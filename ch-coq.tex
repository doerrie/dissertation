\chapter{The Coq Proof Assistant}
\label{ch:coq}

The \TMmodelName{} verification is formalized using the Coq proof assistant.
This chapter begins by motivating Coq as a good candidate for a verification system.
It then presents a brief overview of the term language of Coq with a focus on some the features used in \TMmodelName{}.

\section{Why Coq}

This section discusses the reasons Coq was chosen as the mechanized verifier for \TMmodelName{}.
It describes how higher-order logics can simplify first-order problems by permitting developers to generalize proofs.
It argues that intuitionistic logics are natural choices for automated verification, as truth values are represented as programs inhabiting a type.
It discusses how Coq implements both of these features and articulates the useful features of the Coq standard library for the \TMmodelName{} verification.

The high degree of automation available in first-order systems makes them appear very desirable.
However, the ability to generalize theorems in higher-order proof systems makes them more practical than first-order systems.
Any finite problem can be expressed in a first-order logic.
However, without the ability to generalize theorems, developers quickly become burdened with a multitude of specialized proof obligations.
Many first-order proof systems offer the ability to write functions to generalize these proofs.
However, these functions cannot be verified, increasing the unverified surface of the proof and consequently undermining confidence.
Although the safety problem and confinement policy for capability-based systems can be expressed as first-order problems, this verification effort uses Coq because it is a higher-order proof assistant.

Most higher-order logics available fit into two categories: classical higher-order predicate logic and intuitionistic type theory.
For proofs that can be automatically verified, intuitionistic and classical logics have the same expressive power.
They differ in how they view the concept of truth.
Classical logics reason about truth and falsehood directly, relying upon meta-analysis like term rewriting to prove a theorem.
Intuitionistic logic reasons about construction and refutation directly, requiring a witness for every truth.
These witnesses make statements in intuitionistic logic stronger than classical logic.

The concept of how truth is constructed directly informs computational verification.
Proofs as constructions and proofs as meta-inferences necessarily operate in different ways.
Constructions and theorems have a direct relationship with software terms and types according to the Curry-Howard isomorphism. \msdnote{Not citing}
In Coq, proofs as programs can be directly written by the developer and can be directly manipulated by other programs without resorting to meta-theory, simple type checking may suffice.
This reduces the amount of meta-logic necessary to express properties and forms very natural proofs.

Verifying a theorem in Coq is the act of constructing a program satisfying a type representing the theorem.
To any developer acquainted with parametric polymorphism and (co)inductive data types, reading and manipulating these programs will be familiar.
Base definitions and functions are executable programs that can be readily understood by developers unfamiliar with proofs, reducing mental overhead.
While specifying a program precisely can give the developer a great deal of power, Coq also includes a wide array of tools to help construct programs automatically.
Coq includes a tactic meta-language and pattern-matching system to assist the developer when searching for a program.
These are combined into a hint system that can be combined to produce a high degree of automation for domains that are highly syntactically driven.

The \TMmodelName{} verification utilizes a number of features of the Coq standard library.
Boolean decidability is used to place most propositions into Boolean logic, making them computationally decidable.
Decidability is a critical problem because undecidability hides the unknowable; a theorem with an unknowable assumption can still be verified.
By ensuring that all propositional hypotheses are isomorphic to Boolean functions, \TMmodelName{} ensures that the results are always known.

The Coq standard library also provides meta-theory support for equivalence relations.
The rewrite tactic in Coq can perform automated rewriting of any equivalence relation, not just built-in equality.
It requires only that relevant terms respect the equivalence relation for relevant types.
This permits the model to reasoning about potentially different, but semantically identical, types and terms.

The last major features of the Coq standard library utilized by \TMmodelName{} are the axiom-free finite set and map libraries. \cite{CoqFSet}
These are very large productions modeled after the OCaml Set and Map libraries.
They include a collection of supplemental libraries containing useful theorems pertinent to their interface as well as implementations including a fully concrete definition using lists.

\section{The Coq Term Language}

\Cref{ch:embed,ch:access,ch:safety,ch:flow,ch:confinement} contain a detailed theorem walk-through of the confinement verification in Coq.
As previously stated in \Cref{sect:intro:confidence}, one goal of this dissertation is to produce confidence in the confinement verfification result.
This dissertation presumes that reviewers have confidence in the proof assistant and it therefore does not discuss the mechanics of the proof construction.
As such, this section focuses exclusively on the useful portions of the Gallina term language of Coq.

Gallina is a higher-order functional language and constructive dependent type system that implements higher-order intuitionistic logic.
The syntax and semantics of Gallina are based on OCaml functions and data-types, but with dependently typed parametric polymorphism.
It contains the usual functions, anonymous functions, and fixpoints along with cofixpoints (sometimes called lazy fixpoints).
(Co)Inductive types are a generalized notion of (co)data-type with (co)constructor definitions.
Gallina also includes a highly type-generalized pattern matching system for (co)constructors.

All terms in Coq belong to one of three sorts: \coqkw{Set}, \coqkw{Prop}, and \coqkw{Type}.
\coqkw{Set} is the sort of ``specifications'' or programs, and \coqkw{Prop} is the sort of ``propositions'' or theorems.
The difference between the two is how they handle the type \mbox{\coqvar{*} \(\rightarrow\) \coqvar{*}}.
The type \mbox{\coqkw{Set} \(\rightarrow\) \coqkw{Set}} is not in the sort \coqkw{Set}, but the type \coqkw{Prop} \(\rightarrow\) \coqkw{Prop} is in the sort \coqkw{Prop}.
This makes \coqkw{Prop} impredicative, in that its terms may be self-defining, and \coqkw{Set} predicative, in that it is not.
The sort \coqkw{Type} is stratified and somewhat complex.
\coqkw{Type} is used very little in this effort and may be considered parametric for either \coqkw{Prop} or \coqkw{Set}.

Unlike general programming languages, all functions in Coq are obliged to terminate.
Definitions and anonymous functions do not permit recursion and simply terminate.
Fixpoints permit structural recursion that can be automatically inferred.
General functions permit the developer to specify both the function and the termination measure, though it may be possible for Coq to infer this as well.
Because all functions must terminate, (co)inductive data types are often used to express general propositions as dependent type families.

The connection between programs and proofs can be summarized in relationships with functions and inductive types.
Implication and universal quantification are expressed by dependent function types, respectively with or without a named parameter.
\coqvar{A} \(\rightarrow\) \coqvar{B} is syntactically the same as \(\forall\) (\coqkw{\_}:\coqvar{A}), \coqvar{B} where \coqkw{\_} is any free variable.
\coqvar{True} is the universally inhabited type and \coqvar{False} is the universally uninhabited type.
Negation, written \(\neg\) \coqvar{A} is syntactically \coqvar{A} \(\rightarrow\) \coqvar{False}.

Most other constructions are inductive types or involve pattern matching.
Conjunction, disjunction, and existential quantification are all inductive types.
The type (\coqvar{and} \coqvar{A} \coqvar{B}), written \coqvar{A} \(\wedge\) \coqvar{B}, has one constructor \coqvar{conj} requiring terms of types \coqvar{A} and \coqvar{B}.
The type (\coqvar{or} \coqvar{A} \coqvar{B}), written \coqvar{A} \(\vee\) \coqvar{B}, has two constructors \coqvar{proj1} and \coqvar{proj2} requiring only one term of type \coqvar{A} or \coqvar{B}, respectively.
Existential quantification over a predicate has one introduction constructor, \coqvar{ex\_intro}, that can only be constructed by a witness satisfying the quantified proposition.

\section{Model Abstraction}
\label{sect:coq:modelAbstraction}

\TMmodelName{} is constructed as an abstract implementation to allow it to be used as a framework for future system verifications.
Operating systems are not the only capability-based system; capability-based systems also include virtual machines, language runtimes, and distributed systems.
As an abstract model, \TMmodelName{} focus on the heart of the confinement problem, producing a result applicable across all domains.

\TMmodelName{} utilizes the Coq module type system as an abstraction mechanism.
This decision was motivated by the use of module type abstraction in the axiom-free finite set library.
Because abstractions and axioms are the same structure in Coq, \TMmodelName{} also provides a trivial implementation to produce an axiom-free result.
It is often the case that the abstract module types are verbatim software from implementation modules.
However, it is not possible to produce them by type inference in Coq version 8.3pl2.
As a work-around, the project includes simple tool to syntactically create a module type based on each trivial implementation through very simple annotations.

This verification includes the very primitive Perl script ``typeify.pl'' to automatically produce precise module types from module functors.
It does not process the full language of Coq, but processes the commands line-by-line using a very small state transition routine over a very strict module format.
The module must contain only one internal module functor, declared on a single line, and the functor parameter list must match the module type parameter list.
Theorems are abstracted by replacing the keyword ``Theorem'' with ``Parameter'' and removing all lines between commands ``Proof.'' and ``Qed.''
Although Coq allows theorems to nest and elide the ``Proof'' command, we do not handle these cases.
Two commands are supported as comments to provide better abstraction in the generated types.
The ``(* ABSTRACT *)'' command processes a ``Definition'' into an appropriate ``Parameter'' allowing other modules to override these definitions with other implementations.
The ``(* TYPE\_REMOVE *)'' command eliminates the subsequent line in the generated type altogether and is often used to eliminate helper theorems and lemmas.
Any potential errors introduced by this transformation will be caught when the original module against the generated signature.

It is necessary that each module functor and signature be pure, in that all dependencies are completely captured by parametricity.
The use of existentially declaring a module type loses type information in Coq 8.3 in ways that would be available in Coq 8.2.
Therefore, updated versions of the \COQFMap{} finite map libraries have been constructed to produce appropriate types.
The pattern of constructing pure functors from the trivial implementation is prevalent throughout \TMmodelName{} and produces an axiom-free proof with a type signature that can be satisfied in future efforts,
While this syntactic type construction is used wherever possible, there are certain portions of the proof which are encoded manually.

The following conventions regarding module names are used throughout this dissertation.
Modules that are also files have the \coqfilemodule{FileModule} font face where as inner modules have the \coqvar{InnerModule} font face.
The locations of each inner module should be clear from the surrounding context.
File modules containing functor implementations are suffixed with -\coqfilemodule{Impl}, while modules constructing a fully complete implementation by functor application are suffixed with -\coqfilemodule{Appl}.
File modules containing type signatures, or those which have no abstraction, are given no suffix.
Convenience file modules with supplemental libraries are suffixed with -\coqfilemodule{\_Conv} and are further suffixed as above.
All abstract module types passed as parameters and convenience modules share the same naming convention throughout the proof, which is summarized in \cref{table:coq:moduleConvention}.

\begin{table}
  \centering
  \begin{tabular}{| l | l |}
    \hline
    Instance & Declaration and location \\
    \hline \hline
    \COQARSet{} & \COQFSet{} of \COQAccessRight{} \\
    \COQRef{} & \COQReferenceType{} module of \COQReferences{}.v\\
    \COQRefS{} &  \COQRefSetType{} module of \COQRefSets{}.v\\
    \COQRefSet{} & \COQReference{} \COQFSet{} of \COQRefS{} \\
    \COQEdges{} &  \COQAccessEdgeType{} module of \COQAccessEdges{}.v \\
    \COQAccessGraph{} & \COQAccessGraphType{} module of \COQAccessGraphs{}.v\\
    \COQAG{} & \COQFSet{} of \COQAccessGraphType{} \\
    \COQSeq{} & \COQSeqAccType{} module of \COQSequentialAccess{}.v\\
    \COQCap{} & \COQCapabilityType{} of \COQCapabilities{}.v\\
    \COQCC{} & \COQCapabilityConv{} of \COQCapabilitiesUConv{}.v \\
    \COQCapS{} &  \COQCapSetType{} \\
    \COQCapSet{} & \COQCapability{} \COQFSet{} of \COQCapS{} \\
    \COQInd{} & \COQIndexType{} of \COQIndicies{}.v\\
    \COQObj{} & \COQObjectType{} of \COQObjects{}.v \\
    \COQOC{} & \COQObjectConv{} of \COQObjectsUConv{}.v \\
    \COQSys{} & \COQSystemStateType{} of \COQSystemState{}.v \\
    \COQSC{} & \COQSystemStateConv{} of \COQSystemStateUConv{}.v \\
    \COQSemDefns{} &  \COQSemanticsDefinitionsType{} of \COQSemanticsDefinitions{}.v \\
    \COQSem{} &  \COQSemanticsType{} of \COQSemantics{}.v \\
    \COQSemConv{} & \COQSemanticsConv{} of \COQSemanticsUConv{}.v \\
    \COQExe{} &  \COQExecutionType{} of \COQExecution{}.v\\
    \COQMut{} &  \COQMutationType{} of \COQMutation{}.v \\
    \COQSub{} &  \COQSubsystemType{} of \COQSubsystem{}.v \\
    \hline
  \end{tabular}
  \caption{Declaration and location of common module names.\label{table:coq:moduleConvention}}
\end{table}



