\chapter{System Embedding}
\label{ch:embed}

This chapter discusses how capability-based systems are embedded in \TMmodelName{}.
It begins by presenting how  \TMcaps{} are modeled and then walks the \TMsystemState{} structure, the snapshot of the security state of the system.
It concludes with the operational semantics of the model including semantics for both the security state and potential information flow for each operation.

\section[Names, Access Rights, and Capabilities]{Names, Access Rights, and \\ Capabilities}

\begin{figure}
  \COQReferences{} Module
  
  \COQDOCReferences{}
  
  \vspace{10 mm}
  
  \COQReferencesImpl{} Module 
  
  \COQDOCReferencesImpl{} 
  \caption{\xmakefirstuc{\TMrefs{}} type and implementation.\label{fig:embed:references} \label{fig:embed:indices}}
\end{figure}

There are two forms of names in \TMmodelName{}: names for objects called \term{\TMrefs{}} and names for storage locations within an object called \term{\TMidxs{}}.
Both names must have a total ordering which is both computable and based on equality.
This ordering requirement was chosen to facilitate faster proofs by easing the requirements of the rewrite tactic.
These requirements do not obligate any implementation to use such a strong encoding directly; all names reside behind other abstractions.

The Coq standard library includes modules that directly capture the requirements for names.
The \COQUsualOrderedType{} module from the \COQOrderedTypeEx{} standard library is almost perfect.
An \COQOrderedType{} is a module containing an equivalence, an ordering relation, and a proof of total ordering while a \COQUsualOrderedType{} extends the \COQOrderedType{} to require term equality for the equivalence relation.
Unfortunately, it does not include the tactic hints from \COQOrderedType{} module that assist proof automation.
Therefore, names are built from \COQUsualOrderedTypeWithHints{} module from the \COQOrderedInclude{} module, which reintroduces these hints.
The trivial implementation supplied by \TMmodelName{} uses natural numbers via the \COQNatAsOT{} module, also included in the \COQOrderedTypeEx{} library.

\begin{figure}
  \COQDOCAccessRights{}
  \caption{\COQAccessRights{} module. \label{fig:embed:accessRights}}
\end{figure}

The system contains four access rights: \COQrd{}, \COQwr{}, \COQwk{}, and \COQtx{}, each conferring the ability to perform various system operations.
\Cref{sect:embed:semantics} discusses how these permissions are used in detail, but their intent is simple.
The \COQwr{} permission enables the ability to directly write and overwrite information in the target, or destroy the target altogether.
The \COQrd{} permission allows a subject to read information and capabilities from the target.
The \COQwk{} permission is a sub-type of the read permission that authorizes transitive read-only authority.
Message passing between subjects is authorized by the \COQtx{} permission.

The implementations for most enumerated types in \TMmodelName{} are created by the \COQPOTtoOT{} functor found in the \COQProjectedOrderedType{} module.
This functor produces an \COQOrderedType{} from a \COQProjectedToNat{} module.
Modules satisfying \COQProjectedToNat{} must contain some base type, an injection function from that type to the natural numbers, and a proof of injectivity.
This allows different inductive terms to be quickly defined and used as \COQOrderedType{}s.

\begin{figure}
  \COQDOCCapabilities{}
  \caption{\xmakefirstuc{\TMcaps}. \label{fig:embed:caps}}
\end{figure}

In capability-based systems, a \term{\TMcap{}} is an unforgable binding of a name and permissions.
This is effectively a product of an \TMref{} and a finite set of \TMaccessRights{}.
However, the \TMcap{} abstraction hides such an implementation behind accessor functions and two proofs of equivalence.
Capabilities have the constructor \COQmkCap{} and accessor functions \COQcapTarget{} and \COQcapRights{} with the following properties:
1) injection: two capabilities are equivalent if and only if their accessors have an equivalent \TMcapTarget{} and \TMcapRight{} and
2) equivalence: invoking accessors on the constructor produces equivalent inputs.

The capability type also requires a \NMweaken{} function to produce a weakened form of a capability.
\xmakefirstuc{\TMweakening} a \TMcap{} produces a \TMcap{} with the same \TMcapTarget{}, but eliminates all \TMaccessRights{} except \NMwk{}.
Because the \NMrd{} also entails the \NMwk{} \TMaccessRight{}, a \TMweakened{} \TMcap{} produces \NMwk{} from a \NMrd{}-only \TMcap{}.
The \NMweaken{} function is used by the operational semantics to cause \COQwk{} to be a system-enforced, transitive, read-only \TMaccessRight{}.

\section{Objects and System State}
\label{sect:embed:systemState}

The \term{\TMsystemState{}} represents a snapshot of the permission state in the system at a fixed instant in time.
Structurally, the \TMsystemState{} is a finite map from a reference to a quadruple of an \TMobj{}, \TMobjLabel{}, \TMobjType{}, and \TMobjSchedule{}.
All of these types, with the exception of \TMobjs{}, are enumerations defined by injection to the natural numbers using the \COQProjectedToNat{} type.
An \COQobjLabel{} indicates what phase of the object life-cycle the object is in; it may be \COQunborn{}, \COQalive{}, or \COQdead{}.
The distinction between \COQactive{} \TMobjs{}, such as processes or threads, and \COQpassive{} storage \TMobjs{} is captured by the \COQobjType{}.
The \COQobjSchedule{} is presently a placeholder and will not be used in the remainder of the proof.

An \term{\TMobj{}} represents the permission state of a single object in the system and is a finite map from \term{\TMidxs} to \term{\TMcaps}.
While \TMmodelName{} captures the potential for data motion in the operational semantics, actual data are not represented.
All \TMobjs{} may contain both \TMcaps{} and data.
System implementations are free to implement a partition at object granularity.
\TMmodelName{} makes no guarantee about whether \TMcaps{} are an opaque or visible data structure and conservatively approximates them as visible.\footnote{See \Cref{sect:constructor:capabilitySystems} for more information.}

\begin{figure}
  \COQDOCObjects{}
\caption{\xmakefirstuc{\TMobjs}. \label{fig:embed:objs}}
\end{figure}

\begin{figure}
  \COQDOCSystemState{}
\caption{\xmakefirstuc{\TMsystemState}. \label{fig:embed:systemState}}
\end{figure}

The \TMobj{} and \TMsystemState{} interfaces are a specialization of the \term{\TMOrderedFMap{}} interface, \COQOrderedFMap{}.
In addition to being an \COQFMap{} with keys as an \COQOrderedType{}, an \TMOrderedFMap{} is also required to have an ordering on its value or data.
This allows both of these types to be \COQOrderedType{}s with decidable equivalences over them and permitting them to be nested.
To allow \TMobjs{} and \TMsystemStates{} to also be an \COQOrderedType{}, they include the missing decidable equivalence theorem.
The \TMsystemState{} data type is implemented by composing \COQProductType{} to produce a quadruple.
The trivial implementations use the \COQFMapList{} to construct these types.

As mentioned in \Cref{sect:coq:modelAbstraction}, updates to the Coq module system have caused subtyping issues in the \COQFMap{} library.
It is possible to work around this issue by using pure functors, rather than the type capturing modules.
Unfortunately, while these types exist for the plain \COQFMap{}, they do not exist for \COQOrderedFMap{}.
\TMmodelName{} includes the \COQFMapTwo{}* libraries, which are a modification of the \COQFMap{} libraries, but contain a pure form of the \COQOrderedFMapFun{} type and functors.

Many modules have convenience modules to assist in creating legible results and reusable proofs.
The definitions within these modules will not be discussed until they appear elsewhere and will only be provided if their implementation is not trivial given their description..
There are four convenience modules, one for each underlying module: \COQCapabilities{}, \COQObjects{}, \COQSystemState{}, and \COQSemantics{}.

\section{Operational Semantics}
\label{sect:embed:semantics}

The operational semantics over the system state is defined as the execution of an \TMop{} sequence.
Each \TMop{} is defined in three parts: an \TMop{} precondition, a \TMsystemState{} update function, and an upper bound on information flow between objects.
The semantic model defines the preconditions for each operation and how the system state is altered upon success.
Additionally, each operation contains an upper bound on information flow, for both success and failure.

\begin{figure}
  \COQDOCoperation{}
  \caption{Definition of all \TMops{}. \label{fig:embed:op}}
\end{figure}

There are eight operations that form the operational semantics of \TMmodelName{}.
The \COQread{} and \COQwrite{} operations model only data flow and do not alter the system state in any way, while \COQfetch{} and \COQstore{} operations model capability reads and writes from various objects.
New objects are allocated using the \COQcreate{} operation, and objects are destroyed using the \COQdestroy{} operation.
Executing a \COQrevoke{} command will erase an entry in an object map.
All message-passing protection mechanisms for capability systems are encoded using the \COQsend{} operation.

All \TMops{} have approximately the same structure.
The first parameter of each operation is always the \TMref{} of the invoking subject.
The second parameter, with the exception of \TMcreate{}, is the \TMidx{} of the \TMcap{} being invoked.
In the case of \TMcreate{}, no such capability exists, so the parameter specifies the unborn \TMobj{} being \TMcreated{}.
The \TMidxs{} being accessed and updated are either passed directly or as a list of \TMidx{} pairs when multiple \TMcaps{} may be transferred.
The \TMidx{} option permits a \TMsend{} \TMop{} to fabricate a reply.

These \TMops{} should not be thought of as a system call performed by the invoking subject, or specified in parts by all parties.
They should be considered a specification of which actions the system may take when updating the \TMsystemState{} or shuffling data.
A well constructed system should not ever perform an invalid \TMop{}.
The only reason invalid \TMops{} are defined is to permit \TMops{} and \TMop{} sequences to be pure data structures; the semantics will skip over nonsense \TMops{}.

Each of these operations, with the exception of \COQcreate{}, requires a capability containing an appropriate permission.  
As operations that inspect an object, the \COQread{} and \COQfetch{} operations require \COQrd{} or \COQwk{} access.
The \COQwr{} \TMaccessRight{} is required by the \TMfetch{}, \TMstore{}, \TMrevoke{}, and \TMdestroy{} operations as they modify object state externally.
The \COQsend{} operation is handled with its own permission as it potentially produces a bi-directional relationship via an optional reply capability.
Object allocation through \COQcreate{} is considered universally available and does not require a \TMcap{}. \footnote{Most systems configurations presume processes have the ability to allocate and manage storage.  See \cref{sect:constructor:spacebank}}

\begin{figure}
  \COQDOCPreconditions{}
  \caption{Operation preconditions.  From \COQSemanticsDefinitions{}. \label{fig:embed:preconditions}}
\end{figure}

The definitions for \TMmodelName{} operational semantics are distributed across two modules.
The library \COQSemanticsDefinitions{} contains all of the operation prerequisites and proofs of their decidability.
These proofs are later used in decision procedures in the \COQSemantics{} library when defining operation success and failure.
This separation allows these decision procedures to be included as Boolean decisions in the model signature and implementation.

There are a few omitted definitions from \Cref{fig:embed:preconditions} .
The \COQFNtargetIsAlive{\coqvar{t}}{\coqvar{a}}{\coqvar{s}} function checks that the target of the \TMcap{} at \TMidx{} \coqvar{t} in \TMobj{} \coqvar{a} has \TMobjLabel{} \TMalive{} in \TMsystemState{} \coqvar{s}.
Likewise, \COQFNtargetIsActive{\coqvar{t}}{\coqvar{a}}{s} checks that the \TMobjType{} is \TMactive{} in the same manner.
The \TMsystemState{} convenience library offers the \COQSC{}.\COQFNgetCap{\coqvar{t}}{\coqvar{a}}{\coqvar{s}} function to return an \coqvar{Option} \COQCap{}.\coqvar{t} to find a \TMcap{} at \TMidx{} \coqvar{t} in \TMobj{} \coqvar{a}.
\COQoptionHasRight{} simply shorthand for mapping \COQCC{}.\COQhasRight{} over an \coqvar{Option} \COQCap{}.\coqvar{t} and returning \COQFalse{} in the \coqvar{None} case.

The \COQSemanticsDefinitions{} library is dependent on the \COQSystemStateType{}, and all prerequisites.
It begins by defining a common operation prerequisite \COQFNpreReq{\coqvar{a}}{\coqvar{t}}{\coqvar{s}}, which tests that the agent subject \coqvar{a} is \TMalive{} and \TMactive{} in \TMsystemState{} \coqvar{s}, and that the \TMcapTarget{} of the \TMcap{} at \TMidx{} \coqvar{t} is also \NMalive{} in \coqvar{s}.
All \TMops{} except \TMcreate{} share \COQpreReq{} as a condition on their success.
Additionally, the target capability must have the appropriate access right as previously defined.
The exception to this is the \COQFNcreatePreReq{\coqvar{a}}{\coqvar{n}}{\coqvar{s}} which does not test a target \TMcap{}, but checks that the \TMobj{} \(n\) is \TMunborn{}.
The remainder of the library contains the theorems for the decidability and equivalence of each of these predicates and their support functions.  

\begin{figure}
  \COQDOCValidStateTransitions{}
  \caption{State transitions with preconditions satisfied. From \COQSemantics{}. \label{fig:embed:validStateTransitions}}
\end{figure}

%% \begin{figure}
%%   \COQDOCInvalidStateTransitions{}
%%   \caption{State transitions without preconditions satisfied. From \COQSemantics{}. \label{fig:embed:invalidStateTransitions}}
%% \end{figure}

Valid state transitions and information flow are defined in the \COQSemantics{} library.
Each state transition for operations where preconditions are satisfied is given in \Cref{fig:embed:validStateTransitions}.
The theorems describing the trivial case of no state transition when preconditions are not satisfied have been omitted.
The \NMread{} and \NMwrite{} operations do not cause any state transition in either case and are defined as a single theorem.
The \COQdoOp{} function concretely unifies all of these specific cases together and is specified in \Cref{fig:embed:doOp}.

\begin{figure}
  \COQDOCdoOpSpec{}
  \COQDOCdoOpSpecDoOp{}
  \caption{Specification for \COQdoOp{}. \label{fig:embed:doOp}}
\end{figure}

For brevity, certain definitions have been omitted from \Cref{fig:embed:validStateTransitions}.
\COQoptionMapOne{} and \COQoptionMapTwo{} are specialized \coqvar{Option} structures that combine both mapping and reducing functionality.
Each is applied to a collection of functions of arity zero through one or two, respectively.
Once applied to either one or two \coqvar{Option} types, they apply the number of \coqvar{Some} results to the function of matching arity.
The \TMsystemState{} convenience library also includes the \COQSC{}.\COQcopyCap{} and \COQSC{}.\COQweakCopyCap{} functions to perform a standard or weakening \TMcap{} transfer.
The \COQSC{}.\COQcopyCapList{} function performs many such copies using an \TMidx{} pair list as a map.
All \TMcaps{} with a specific target are removed from the system using \COQSC{}.\COQrmCapsByTarget{}.
\COQSC{}.\COQupdateObj{} updates only the \TMobj{} at a reference and \COQSC{}.\COQsetAlive{} or \COQsetDead{} updated the \TMobjLabel{} appropriately. 

\begin{figure}
  \FIGflow{}
  \caption{Reproduction of information flow from \Cref{fig:sketch:flow}.\label{fig:embed:flow}}
\end{figure}

\begin{figure}
  \COQDOCreadFromSpec{}
  \COQDOCreadFromDef{}
  \caption{Definition of \COQreadFrom{}. \label{fig:embed:readFrom}}.
\end{figure}

\begin{figure}
  \COQDOCwroteToSpec{}
  \COQDOCwroteToDef{}
  \caption{Definition of \COQwroteTo{}. \label{fig:embed:wroteTo}}
\end{figure}


The potential for data motion is captured by the \COQreadFrom{} and \COQwroteTo{} definitions.
During an operation, data may flow from any \TMobj{} in the \COQreadFrom{} set to any \TMobj{} in the \COQwroteTo{} set.
The specification of these functions is given in \Cref{fig:embed:readFrom,fig:embed:wroteTo}.
However, \Cref{fig:sketch:flow} summarizes the conditions simply, and it is reproduced in \Cref{fig:embed:flow}.
Self-mutation is not modeled by any \TMops{} and \TMmodelName{} presumes all \TMactive{} objects may alter their own state.

\begin{figure}
  \COQDOCexecuteSpec{}
  \COQDOCexecuteDef{}
\end{figure}

Performing an \TMop{} sequence is modeled by the \COQexecute{} function applied to an \TMop{} list, provided by the \COQExecution{} module.
Because the semantics will ignore any invalid \TMop{}, it is defined as simply performing the \TMops{} sequentially.
The only curious property of the \COQexecute{} function is that it performs operations on a list in reverse.
This is done to align list induction with execution induction.
An empty list performs no \TMops{} and, given a \TMsystemState{} that is the result of \TMexecuting{} an \TMop{} list, performing another \TMop{} is the result of the \coqvar{cons} constructor and not an \coqvar{append} function.
