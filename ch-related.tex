\chapter{Related Work}
\label{ch:related}

This chapter discusses other access-control proofs that are related to \TMmodelName{}.
Constructing confinement from memoryless subsystems is an approach that differs from the constructor model.
The Provably Secure Operating System, PSOS, was the first attempt to specify and build a capability-based operating system that supported confinement through memoryless subsystems.
SCOLL is a language for describing the behavior of entities within capability-based systems, and to reason about different authority configurations the system may come to have in the future.
seL4 is the first microkernel to be fully connected to its specification by automated verification.
The take-grant models for seL4 are similar to \TMmodelName{}.

\section{Memoryless Subsystems}

Memoryless subsystems are a protection mechanism for mutually suspicious programs.
In ``Memoryless Subsystems,'' \cite{Fenton:MemorylessSubsystems} Fenton constructs a general-purpose machine with infinite registers that are statically tagged as privileged or unprivileged, with the exception of the program counter.
The program counter has a variable tag which is upgraded to privileged whenever impacted by privileged data.
The machine provides a call-return procedure that preserves and restores the value and protection-state of the program counter and the machine may only halt in the unprivileged state.
From this, he defines a memoryless system as one that, when started in an unprivileged state, no unprivileged information may be impacted by privileged information.
His solution to leaking information by non-termination requires both privileged and unprivileged input to provide counters for their respective operations.

The use of memoryless subsystems to produce security policies depends on their initial configuration and how resulting data is dispersed by the system.
Memoryless subsystems can be used as part of a confinement model that differs from the constructor model.
A memoryless subsystem where all output is considered privileged must also be confined, assuming the destination of this privileged data is authorized by the caller.
Enforcing this constraint upon a memoryless subsystem can be accomplished by erasing all non-privileged registers after execution or by requiring all registers to be privileged.

While it is possible to model memoryless confinement in \TMmodelName{}, it is difficult because \TMmodelName{} does not consider state as part of the model.
Therefore, each invocation of a memoryless subsystem must be modeled in \TMmodelName{} as a fresh subsystem over shared unprivileged state, if any.
This violates the intuition that there is an injective map from objects in the system to objects in the model.

\section{PSOS}

The Provably Secure Operating System, PSOS, was the first attempt at a mechanically verifiable operating system built on sound design principles \cite{PSOS:TR}.
The PSOS specification and implementation is structured as a layered TCB.
Each layer is clearly delineated and refines the system by adding services and features not present in previous layers.
The lowest layers of the system are constructed from computational hardware and the minimal software to construct capabilities.
Subsequent layers range from memory management and I/O up to users and applications.
Many high-level structures in modern capability-based systems resemble those developed as part of the PSOS specification.

The PSOS system specification was not only rendered in English text, but also in the non-procedural specification language SPECIAL.
SPECIAL is an assertion language designed to capture system requirements and to verify their composition as part of the refinement process.
Many theorems about the different layers of PSOS were verified in SPECIAL.
The SPECIAL project would eventually be succeeded by the PVS \cite{cade92-pvs} proof system.

In PSOS, the Confined System Manager, or CSM, is responsible for constructing confined subsystems.
As with the constructor, the CSM tests the specification of an object before instantiating it.
Confinement provided by the CSM requires all objects to be recursively memoryless so that they may be safely reused by otherwise independent subsystems.
This permits the CSM to memoize the result of producing confined subsystems to aid in confinement attestation and to permit complex structures to be built through repeated invocation.

When an application desires to make a confined entity, it presents a specification of that entity as a vector of capabilities.
The {\tt make\_segments\_confined} interface inspects all elements for capabilities to known confined system functions, previously certified objects, or another element of the presented vector.
If satisfied, the CSM retains a copy of all capabilities and returns a copy of the vector where all write or delete access has been removed.
These capabilities include the ``confined-delete'' permission, which can be used in conjunction with the CSM to perform deletes only to certified segments.
To make extended objects, the CSM essentially has the same behavior, though the capability vector upon which the object is based is not granted ``confined-delete'' permissions.
This restriction is in place as a confined object must not be able to delete capabilities from another confined object as this could cause data transfer.
Objects not so constrained may write data to a confined subsystem through deletion without violating confinement.

Many theorems of high-level PSOS properties remain unpublished or unverified.
Unfortunately, this includes the confinement property for the Confined Systems Manager.
As it is using a memoryless confinement mechanism, the PSOS CSM can be conceptually modeled by \TMmodelName{} using fresh subsystems.

\section{SCOLL}

SCOLL, the Safe Collaboration Language, is a system description language for statically reasoning about the authority relationships of subsystems in capability-based systems. \cite{Jaradin05scoll}
SCOLL allows developers to define a system, its semantics, an initial configuration, Horn clauses defining the behavior of various subsystems, and goal conditions which are to be preserved.
Static reasoning is performed by the SCOLLAR model checker, which computes constraints that preserve the goal conditions while attempting to be minimal.

Alfred Spiessens presents a proof of the safety property for capability-based systems as part of his work on SCOLL.\cite{Spiessens07patternsof}
SCOLLAR relies on the tractability of the safety property to produce solutions which preserve the goal conditions.
Positive goal conditions are \term{liveness} constraints, while negative goal conditions are \term{safety} constraints.
Although SCOLL and SCOLLAR are not formally verified, they do produce meaningful, and more importantly, understandable solutions to the safety problem.

SCOLLAR has been used to examine and validate a number of capability-based patterns.
Revocable capabilities via the caretaker pattern has been examined along with the methods by which it can be circumvented.
Inescapable interposition has also been examined.
In this scenario, two fully untrusted subjects are prevented from escalating their authority to one another by the interposition of proxies with known behavior.
SCOLLAR has determined that no further behavioral restrictions are necessary to preserve the interposition arrangement, indicating that escalating authority is impossible.

SCOLL is an extremely useful tool for reasoning about safety in capability-based systems.
However, SCOLL has not been extended to reason about information flow and therefore cannot describe the confinement problem.
Due to their legibility, solutions presented by SCOLLAR are easily inspected and understood.
However, the safety property upon which SCOLLAR relies has not been mechanically checked.

\section{seL4}

\Cref{sect:corr:seL4} presents an brief overview of the seL4 system.
As previously mentioned, the seL4 microkernel \cite{Elphinstone:formalisingukernel} is the L4 microkernel \cite{Liedtke:1996:TRM} implementation modeled and inspected as part of the L4.verified project \cite{Klein:l4.verified}.
To achieve this goal, the seL4 mirokernel is a capability-based system.

In contrast to the systems described in \Cref{ch:constructor}, memory management interfaces in seL4 are handled directly by the kernel.
seL4 provides \term{Untyped Memory} objects representing system memory.
An Untyped Memory object may be subdivided into other Untyped Memory objects or into other primitive kernel objects.
The kernel maintains a \term{Capability Deriviation Tree}, or CDT, to track the relationships between capabilities naming object derivations.
When an Untyped Memory object is retyped, the system checks the CDT to determine which capabilities to invalidate.
This interface is extended to all capabilities, and processes may choose to either \term{copy} capabilites, creating siblings in the CDT, or by \term{minting} new and restricted capabilities as children in the CDT.
The terms \term{copy} and \term{mint} sometimes appear as \term{imitate} and \term{grant} in the seL4 literature.

\subsection{seL4 models}
\label{sect:sel4:models}


Elkaduwe et. al. presented the first access control model for seL4. \cite{seL4:Verified} \cite{Elkaduwe:Thesis}
Like \TMSW{} and \TMmodelName{}, it is a state-transition model over a capability graph as a system state.
All operations require a \term{sane} system state in which all capabilities name objects that are part of the system.
The operation preconditions require a sane state and appropriate capabilities authorizing the operation.

The Elkaduwe model partitions data and capability operations based on access rights following earlier take-grant models \cite{LiptonSnyder:LinearSecurity} \cite{Snyder:1977:SAP}.
Data-only operations require \term{Read} or \term{Write} permissions, while all state update operations are modeled with \term{Grant} and \term{Create}.
The ability for a thread to retrieve a capability from a storage object is modeled by inverting the relationship so it appears that the storage object \term{Grant}s capabilities to the thread.
The ability to create new objects requires holding a \term{Create} capability to some other object, presumably representing untyped memory.

The operations of the system are straight-forward.
\term{SysNoOp} represents any self-mutation not involved with a capability.
\term{SysRead} and \term{SysWrite} model explicit data mutation and require capabilities with \term{Read} or \term{Write} permissions respectively.
\term{SysGrant} requires two capabilities: the invoked capability with the \term{Grant} permission and a capability to transfer.
The \term{SysGrant} operation may ``diminish'' the transferred capability by reducing the available access rights.
\term{SysCreate} also requires two capabilities: the invoked capability with the \term{Create} permission, and a \term{Grant} capability to an object where a capability to a new object will be inserted.
\term{SysRemove} removes a specific capability from an object using a capability, and \term{SysRevoke} revokes an unspecified collection of capabilities from the system by repeatedly invoking \term{SysRemove} based on the capability derivation tree.

The model defines \term{subsystems} as an emergent property of the arrangement of capabilities.
First, the \term{leak} judgment is defined from the holder of a \term{Grant} capability to its target.
Next, two objects are \term{connected} if there is a \term{leak} between them in either direction.
Finally, two objects are members of the same \term{subsystem} if they are in the transitive, reflexive closure of \term{connected}.
They then verify that, from this point, it is not possible for the access rights between members of subsystems to increase over the execution of the system.
This must be the case as both operations to add capabilities to the state of the system require the \term{Grant} permission.

The ability for information to flow between objects is examined next.
The \term{canAccess} judgment is defined between objects \(A\) to \(B\) if any capability containing a subset of \(\{Grant, Read, Write\}\) access rights exists between them in either direction.
All information flow between objects is defined as the transitive, reflexive closure of \term{canAccess}.
As before, this initial property is verified to hold over the life of the system.
Therefore, the potential information flows over the life of the system is statically determined.

There are two problems with the model.
First, the definitions of subsystems and authorized flows are emergent rather than intentional.
That is, subsystems emerge as those collections of objects which can share capabilities irrespective of any human labeling or intent in their construction.
This is not the standard definition of a subsystem.
A subsystem is any subset of the system objects to be considered for inspection.
By restricting subsystems in this way, many desirable properties, including confinement, become impossible to express.
The same problem arises with information flow.
While information flow may not increase between emergent subsystems, there is no discussion of whether that information flow was expected or how to constrain it without complete partitions.

The other problem with the model is that no access rights are required to remove capabilities from an object.
Recall that capabilities are data and \emph{capability flow is data flow}.
Both \TMSW{} and \TMmodelName{} ensures that all potential data flow are modeled as such.
Unfortunately, no access rights are required to perform either \term{SysRemove} or \term{SysRevoke} in the Elkaduwe model.
This creates an information channel when capabilities are deleted from objects without holding a \term{Write} or \term{Grant} capability.

In the case of \term{SysRemove}, a capability with no permissions still authorizes data to be written.
The deletion of a capability can be observed by a subject through the attempt to use a now defunct capability.
This could be caused by attempting to read from a page no longer in virtual memory, or by attempting to invoke a capability no longer in a CNode.
Because this information flow is not captured by a \term{SysWrite} operation, it is not modeled as data motion.

The capability derivation tree, or CDT, is modeled as an axiom that also hides information flow at a much grander scale.
The capability derivation tree is a map from capabilities to their parent.
When the parent capability is revoked, all children are also revoked.
The CDT is not modeled as part of the system and is assumed as a function that returns some unspecified collection of capabilities to remove.
Performing \term{SysRevoke} could remove any capability in the model; it follows that performing \term{SysRevoke} may write data to \term{as many as all objects in the system.}

The CDT in the seL4 \emph{implementation} is not so unconstrained.
However, the issue can still arise as the revocation of a capability will delete all derived child capabilities.
If a system is constructed such that a child capability crosses an intended subsystem boundary, then it is possible that an unauthorized communication channel between subsystems exists.

These flaws could have been detected if the Elkaduwe model had described information flow occurring at each operation, as is the case in \TMSW{} and \TMmodelName{}.
If the \term{SysRemove} and \term{SysRevoke} operations had been modeled with data motion, then proofs of consistency between the permission state of the model and the information flow by operations would not have been possible.
Because the system is finite, such a proof would often contain enough information to infer the conditions for a counterexample, namely when no permissions are present.
This would obligate an update to the model, presumably requiring a \term{Grant} permission to perform the  \term{SysRevoke} and \term{SysRemove} operations.

These specification flaws highlight two subtleties arising when building a system model.
Deletion, even deletion of access control permissions, is a form of data flow and can be easily overlooked when building a system model.
Using emergent behavior to define expected behavior \emph{weakens} a model.

The Boyton model is the second take-grant model developed for seL4. \cite{Boyton_ssv2009}
It's primary contribution is to extend the Elkaduwe model to reason about shared capability storage.
Like most capability-based systems, seL4 uses capabilities to build high-level structures such as address spaces and applications.
Some of these structures may be understood by the kernel itself, as is the case for address spaces.
This causes some interpretations of the security infrastructure to appear to modify objects from a distance, such as threads modifying pages or capability-lists through an address space.

The solution presented by the Boyton proof is to introduce a \term{store} permission into the system, which is required to place a capability into thread storage.
Object \(A\) is \term{directly store-connected} to \(B\) if \(A\) holds a capability with \emph{Store} permission naming \(B\).
Objects are \term{store-connected} by transitive, reflexive closure over direct store connections.
When examining the capabilities which may be used by an object during any operation, all capabilities accessible in store-connected objects are considered.
The \term{SysGrant} operation is updated to write to any store-connected objects and the \term{SysCreate} operation returns a \(\{Grant,Store\}\) capability.
Taken together, these admit loads and stores through memory traversal to be captured by a single model operation.

The definition of \term{leak} is nearly unchanged in the Boyton proof.
Because shared storage is another mechanism of transferring capabilities, any two objects that are store-connected to a third can \term{leak} capabilities and are therefore \term{connected}.
The definition of a \term{subsystem} is still emergent as those collections of objects that are not connected.
As with the Elkaduwe proof, this system can only reason about emergent mandatory properties and cannot reason about security policies, such as confinement.
Additionally, no change to capability revocation is made.
Capability deletion and the CDT remain a back-channel of information flow.

The Boyton proof does refine the notion of information flow.
The definition \term{set-flow} is defined as \(A\) can flow to \(B\) if there is a capability with \term{Write} from \(A\) to \(B\) or a capability with \term{Read} from \(B\) to \(A\).
\Term{flow} is the transitive, reflexive closure of \term{set-flow}.
This two-stage emergent behavior is slightly more refined than the \term{canAccess} definition of Elkaduwe as it does not consider all connections bidirectional.
However, this definition only discusses data reads and writes that could occur via the read and write operations excluding all others.

It should be noted that these models have never been and cannot be connected to the current security proofs of seL4.
The policies that have been verified are specified on a per-subject basis and remain low-level.
The policy developer must have deep knowledge of primitive system structures to successfully define a policy. 
The policies that can be described are global, mandatory, and cannot describe dynamic border policies, such as confinement.
Extending these policies to a high-level of abstraction remains an unsolved problem.

\section{CHERI}

CHERI is a capability-based instruction set architecture (ISA) and compiler that provides hardware support for the simultaneous execution of legacy and capability-based software. \cite{watson2015cheri}
This support is extended to link legacy and capability-based software libraries at either the source or binary level and does not require a transition away from the C programming language.
Offering a hybrid environment is intended to facilitate the transition of legacy TCB software into capability-based software.

The CHERI ISA is an extension of the 64-bit MIPS RISC ISA extended with capability support.
The CHERI CPU is implemented as an FPGA soft-core processor with a standard MMU and capability co-processor.
When the capability co-processor is disabled, CHERI can run MIPS operating systems without modification.
A system supervisor can enable the capability co-processor to provide capability-based protection within applications.

CHERI capabilities are segment descriptors that also include permissions and a type field.
Holding a CHERI capability grants the specified access to a range of memory defined at byte-granularity.
Unlike IA32 segments, capability address translation is applied \emph{before} MMU virtual address translation with the effect that capability-based segments are internal to each MMU-based address space.
The CHERI ISA permits both capabilities and data to reside in the same MMU-based address space and maintains a hardware-enforced separation between them.
Capabilities are tagged by a capability bit and, whenever a capability is modified by a data instruction, the capability bit is cleared and can not be recovered.
Instructions modifying capabilities are always monotonically non-decreasing over the segment range and permissions.

Capabilities can be used either explicitly or implicitly in CHERI.
They can be loaded and stored in the capability register file, or referenced and dereferenced in memory.
Therefore, CHERI capabilites can be used as safe pointers by C compilers similar to Cyclone's fat pointers. \cite{Jim:2002:CSD}
``Global'' capabilities are universally accessible in an MMU-based address space, and can be used to provide legacy software and libraries access to specific portions of an application heap.

CheriBSD is a modification of FreeBSD to support hybrid application compartmentalization using CHERI.
Like a traditional Unix, each application is given its own MMU-based address space.
Capability-based compartmentalization is primarily managed via the construction of a per-thread trusted stack that links a chain of disjoint per-compartment stacks.
Invocations of \term{CCall} (or resp. \term{Return}) push (or pop) a capability to a new (or the previous) compartment stack to (or from) the trusted stack to preserve compartmentalization.
CheriBSD also provides ABI wrappers to link capability-based software libraries with legacy libraries within an application, and vice versa.
These wrappers come in two varieties with different goals: one enforcing the greatest compartmentalization and one enforcing the most compatibility.
To support the use of capabilites when communicating with the system, CheriBSD provides capability-based wrappers for many system objects.

\TMmodelName{} models pure capability-based systems and is not intended to model hybrid systems like CheriBSD.
Although CheriBSD guarantees compartmentalized code is capability-based and limits legacy access within compartments, it necessarily implements standard Unix mechanisms for legacy applications and libraries.
These applications are not bound by capability-based access control, making cross-compartment policies difficult to implement.

\TMmodelName{} can be applied to a capability-based system running on CHERI, but may also be applicable to a compartment running in CheriBSD.
Compartments in CheriBSD may be made purely capability-based, even when linked against legacy code using an ABI wrapper.
This might be accomplished by a compartment that does not permit global capabilities and only links legacy libraries fully sandboxed using capabilities.
All access within such a compartment must be governed by the use of capabilities, including access outside the system.
Provided no system object can be used to fabricate capabilities, the constructor pattern can be applied by a one-to-one correspondence with \TMmodelName{} objects and operational semantics.
