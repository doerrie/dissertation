\chapter{Information Flow}
\label{ch:flow}

This chapter focuses on analyzing how the flow of data in \TMmodelName{} is related to \TMpotAcc{}.
As previously mentioned, the model does not directly capture application data; the transfer of application data is abstracted using the \NMreadFrom{} and \NMwroteTo{} definitions.
From these definitions, \TMmodelName{} defines what is \term{\TMmutated{}} through a sequence of operations.
It also supplies the \term{\TMmutable{}} judgment: a simple, permission-based definition of mutability by directly examining an access graph.
The remainder of the chapter examines how what is mutated through a sequence of operations is a subset of what is considered potentially mutable.
Naively, this is demonstrated by observing that all information flow occurs by the existence of a capability and that, for each operation, what is mutated is a subset of what is mutable in the direct access.
By capturing all potential capabilities, the computation of potential access has also captured all potential information flow.

\section{Mutation}

The semantics of \TMmodelName{} do not incorporate data values directly.
Instead, they define how data may flow during each operation through the use of the \COQreadFromSpec{} and \COQwroteToSpec{} relations.
As their names imply, the \COQreadFromSpec{} judgment indicates the potential information flow sources during an operation, and \COQwroteToSpec{} indicates the potential destinations.
A complete system implementation must demonstrate that its operations do not violate these judgments, providing implementations of \COQreadFrom{} and \COQwroteTo{} satisfying those specifications.

For review, the definition of \COQreadFromSpec{} and \COQwroteToSpec{} capture potential flows of both data and \TMcaps{}.
The simple \NMread{} and \NMwrite{} operations that have no impact on the \TMsystemState{} indicate data-only data motion.
Because \emph{it is possible to encode data using \TMcaps{}}, their potential transfers are considered as information flow.
While necessary for correctness, this is not the case for all models and many incorrectly label such flow as \emph{covert}.
Thus, the \TMfetch{} and \TMstore{} operations admit the same mutability as \COQread{} and \COQwrite{}.
\xmakefirstuc{\TMsending{}} a message may contain both data and \TMcaps{}, and has the same mutation as \COQwrite{}.
Like the \COQsend{} \TMop{}, \TMcreate{} is able to pass multiple \TMcaps{} and arbitrary data along for \TMobj{} instantiation.
The \COQrevoke{} \TMop{} has the same flow potential as \COQwrite{}.  In many systems the \TMrevoke{} command is performed by a \COQwrite{} of a trivially non-mutating \TMcap{}.
Though it alters the state of an \TMobj{}, the \COQdestroy{} \TMop{} is not modeled as effecting mutation upon that \TMobj{}.
When an \TMobj{} is destroyed, subsequent \TMops{} on that \TMobj{} do not admit any outward information flow, not even the state of the \TMobj{}.\footnote{This is hiding a subtle issue and is addressed in \Cref{ch:future}.}

\begin{figure}
  \COQDOCmutatedOpDef{}
  \COQDOCmutatedDef{}
  \caption{Definition of \COQmutated{}. \label{def:flow:mutated}}
\end{figure}

The \COQmutatedDef{} inductive models where information might flow during a sequence of operations.
Specifically, \COQmutatedDef{} considers the \TMobjs{} reachable by an initial subsystem in an initial \TMsystemState{}.
As a precondition, the initial subsystem must contain \TMobjs{} that are \TMalive{} or \TMdead{}.
The algorithm is initialized with the initial subsystem being the reachable subsystem, as all \TMobjs{} are considered self-mutating.
Each time an \TMop{} is performed the reachable subsystem is expanded to include the \COQwroteTo{} set if an element of the subsystem is present in the \COQreadFrom{} set.
After all \TMops{} have been executed, the resulting reachable subsystem contains all locations information in the initial subsystem could have reached as a result of that \TMop{} sequence.

\section{Mutability}

\begin{figure}
\COQDOCmutableSpec{}
\caption{Definition of \TMmutable{}. \label{def:flow:mutable}}
\end{figure}

In contrast to the flows that occur over operations, the abstract concept of possible mutation as authorized by \TMaccessRights{} is captured by the \COQmutable{} judgment.
Intuitively, most developers expect the potential for information transfer to be expressed using system permissions.
A definition of \COQmutable{} that relies on \COQmutated{} would defeat such intuitions as they have been captured by \TMpotAcc{} and the safety property.
Therefore, \COQmutable{} is defined simply by inspecting the permissions of an access graph, without considering possible operations.
If any \TMobj{} in the initial \TMsubsystem{} is the \TMedgeSource{} of a \NMwr{} or \NMtx{} \TMaccessEdge{}, the \TMedgeTarget{} is \TMmutable{} by the \TMsubsystem{}.
Additionally, if the \TMedgeTarget{} of a \NMrd{} or \NMwk{} \TMaccessEdge{} is in the initial \TMsubsystem{}, then \TMedgeSource{} is \TMmutable{} by the \TMsubsystem{}.

\begin{figure}
  \COQDOCmutableSpecSubset{}
  \COQDOCproperMutableSpec{}
  \caption{Properties of \TMmutable{}. \label{fig:flow:mutableProperties}}
\end{figure}

This definition of \TMmutable{} capturing the instantaneous information flow of each \TMaccessRight{} within any \TMaccessGraph{}.
\xmakefirstuc{\TMmutable{}} is monotonic over the \TMaccessGraph{}, following from the intuition that increasing \TMaccessRights{} increases \TMmutability{}.
This allows all proofs describing approximations over \TMaccessGraphs{} to be lifted to approximate \TMmutability{}.
The fact that a \TMsubsystem{} can self-mutate is captured directly.
\xmakefirstuc{\COQmutable{}} is also monotonic over the initial \TMsubsystem{}, causing it to be non-decreasing.
Applying \TMmutable{} to a \TMdirAccAG{} is called \TMdirAccMutability{} while applying it to \TMpotAccAG{} is called \TMpotAccMutability{}.
Because \TMdirAcc{} is a subset of \TMpotAcc{}, \TMdirAccMutability{} is a subset of \TMpotAccMutability{}.

\section{Operational Mutability}

The hypothesis that all direct mutation is captured by direct mutability and all potential mutation by potential mutability does not match the definition of \TMmutable{} because it does not capture any transitivity of information flow.
For \TMaccessGraph{} approximations to successfully capture \TMmutation{}, the opportunity to capture the transitivity of flow must occur when approximating \TMpotAcc{}.
Naively applying \COQmutable{} after each \TMpotAcc{} approximating operation produces an induction strategy based on \TMpotTransfer{}.
However, the induction strategy of \TMpotTransfer{} is insufficient to describe how \TMmutability{} grows from the initial \TMsubsystem{}.
Because \COQmutable{} is a static definition and does not describe how \TMmutability{} changes over \TMops{}, it cannot approximate \COQmutated{}.
A definition of \TMmutability{} that follows the induction strategy of \COQmutated{} is needed.

\begin{figure}
  \begin{center}
    \FIGgeneralApproxDirAccDep{}
  \end{center}
  
  \COQDOCgeneralMDirAccDep{}
  \COQDOCapproxDirAccDep{}
\caption{Figure and definition for \COQgeneralApproxDirAccDep{}. \label{fig:flow:generalApproxDirAccDep}}
\end{figure}

\Term{\TMmutableExecute} is the concept of folding \COQmutable{} over an \TMop{} sequence in the same manner as \COQmutated{}.
However, because \TMmutableExecute{} must be general enough to extend to all partial \TMpotTransfers{} approximating \COQdirAccExecute{}, it has a slightly complicated structure.
\COQgeneralApproxDirAccDep{} is constructed as a generalized form of \COQpotAccApproxDirAccDep{} that relates two \TMaccessGraph{} transformations by the function \coqvar{Fg}.
\COQpotAccApproxDirAccDep{} becomes the special case where \coqvar{Fg} is \COQpotAccFun{} and the related transformations are \COQdirAccExecute{} and \COQpotAccExecute{}.
\COQapproxDirAccDep{} places additional constraints on \COQgeneralApproxDirAccDep{}.
It requires all functions to be equivalence preserving and restricts \coqvar{Fg} to also be non-decreasing; these conditions are satisfied by all \TMpotTransfers{} and approximating functions.

\begin{figure}
  \COQDOCindexed
  \COQDOCmutableExecute
  \caption{Operational mutability. \label{fig:flow:mutableExecute}}
\end{figure}

\Term{\TMmutableExecute}, defined by \COQmutableExecute{} in \Cref{fig:flow:mutableExecute}, folds \COQmutable{} over each \TMop{} to aligns with the induction strategy of \TMmutated{}.
Unlike the static definition of \TMmutable{}, operational mutability is defined inductively over a list of \TMops{}.
Like \COQmutated{}, it starts with an initial system state and subsystem, and grows these based on what is statically \TMmutable{} after each step.
However, rather than determining what was \TMmutated{}, it is parameterized over a specification for approximating \COQdirAccExecute{} and a function relating this specification to \COQdirAccExecute{}, using \COQapproxDirAccDep{}.
Because the executable specification is a relation between \TMop{} lists and functions, the \COQindexed{} constraint ensures that all equivalent lists are related to equivalent functions.
As will all other general properties, this is a fairly trivial property for all relevant functions.


\begin{figure}
  \COQDOCProperMutableExecute{}
  \COQDOCmutableExecuteSubset{}
  \caption{Properties of operational mutability. \label{fig:flow:properMutableExecute}\label{fig:flow:mutatedExecute}}
\end{figure}

Operational Mutability has two very useful properties: it preserves \TMaccessGraph{} equivalence and subset relationships.
This generalization applies to any execution sequence in a step-wise approximating relationship with \COQdirAccExecute{}, including all \TMpotTransfer{} approximations including \COQpotAccExecute{}.
The proof that the operational mutability of \TMdirAcc{} approximations must be bounded by the operational mutability of \TMpotAcc{} approximations involves specializing these theorems correctly.
This result forms the foundation of information flow analysis for the remainder of this chapter.

\begin{figure}
  \COQDOCmutableDirAccExecute
  \COQDOCmutablePotAccExecute
  \COQDOCexistsMutableDirAccExecute
  \COQDOCexistsMutablePotAccExecute
  \COQDOCmutableExecuteDirAccSubsetPotAcc{}
  \caption{Specialized forms of \COQmutableExecute{}. \label{fig:flow:mutableExecuteSpecializations}}
\end{figure}

Operational mutability is specialized for \TMdirAcc{} approximations using the identity function and for \TMpotAcc{} approximations using \COQcomputePotAcc{}.
Specialized forms are constructed by simple instantiation and checking completeness.
For completeness, the proof that the \TMmutability{} of \COQpotAccExecute{} approximates the \TMmutability{} of \COQdirAccExecute{} is also instantiated.

\pagebreak[4]
\begin{figure}
  \centering
  \FIGmutableExecute{}
  \caption{Relationships between mutated and operational mutability. \label{fig:flow:mutatedMutable}}
\end{figure}

\section{Mutation is always Mutable}


\Cref{fig:flow:mutatedMutable} illustrates the relationship between \TMmutated{} and operational mutability as one induction strategy.
This diagram introduces some additional notation.
Sets of \TMrefs{} are depicted with drop-shadow diamonds for both \TMmutated{} and \TMmutable{} sets.
Also, the figure contains dashed lines to indicate portions of \TMmutableExecute{} that are not part of its definition, but are intended to connect to the definition.
The diagram has been flattened to appear sequential when this is not actually the case.
All functions are actually parameterized over the initial \TMsystemState{} and \TMdirAccAG{} and capture all possible functions approximating the whole \TMop{} sequence.
Though slightly overspecialized, the resulting diagram is more intuitive and more closely aligns with other diagrams in this document.

\xmakefirstuc{\TMmutablePotAccExecute} approximates \TMmutableDirAccExecute{}.
To show that \TMmutablePotAccExecute{} approximates what is \TMmutated{}, it is sufficient to show that \TMmutableDirAccExecute{} approximates what is \TMmutated{}.

\begin{figure}
\COQDOCmutableDirAccExecuteMutatedApprox{} 
\COQDOCmutatedApproxMutablePotAccExecute{} 
\caption{Approximations of Mutated \label{fig:flow:mutatedapprox}}
\end{figure}

It must be the case that \TMmutableDirAccExecute{} approximates \TMmutated{} by induction on \TMops{}.
Presume the \TMsubsystem{} analyzed for \TMmutableDirAccExecute{} is a superset of that analyzed for \TMmutation{}.
The prerequisite \TMcap{} for each \TMop{} is represented as a collection of \TMaccessEdges{} in the \TMdirAccAG{}.
By case analysis, the \TMmutability{} of these \TMaccessEdges{} directly capture the information flow possible for this \TMop{}.
However, \TMmutableDirAccExecute{} grows its subsystem by all possible \TMmutations{}, producing a superset of what was \TMmutated{}.
In the base case, nothing is \TMmutated{} and the \TMmutableDirAccExecute{} is simply direct \TMmutability{}, which is a superset of the initial \TMsubsystem{}.
Therefore, \TMmutableDirAccExecute{} approximates \TMmutated{}, and this result must extend to \TMmutablePotAccExecute{} using previous results.

\begin{figure}
\COQDOCAGProjectMaximal{} %  - AG\_project on a maximal accessgraph is also maximal.
\COQDOCmutableMaximal % - mutable on a maximal access graph does not change.
\COQDOCmutableProjectNotInEq % - operational mutability on maximal accessgraphs does not change through a projection where the allocator is not in the mutable set.
\COQDOCmutableProjectInEq % - operational mutability on maximal accessgraphs grows precicely by the allocated node when the allocator is within the mutable set.
\caption{Relationships of mutable in potential access approximations. \label{fig:flow:AGProjectMaximal}\label{fig:flow:mutableExecuteProp}}
\end{figure}

The remainder of this section describes how initial potential \TMmutability{} and \TMmutablePotAccExecute{} are related.
Specifically, the goal is to demonstrate that initial potential \TMmutability{} for existing \TMobjs{} will never change and subsumes all \TMmutablePotAccExecute{}.
This analysis begins by examining the relationships of \TMpotAcc{} approximations, as shown in \Cref{fig:flow:mutableExecuteProp}.
First, \COQmutable{} is a \TMmaximal{} result when applied to a \TMmaximal{} \TMaccessGraph{}, as shown in \COQmutableMaximal{}. 
That is, each \TMpotAccAG{} has captured all potential information flow for existing objects and operational mutability does not grow for operations that do not alter \TMpotAcc{}.
Since \COQmutable{} preserves subset, it also can't be any smaller and must therefore be equal.
Second, operational mutability grows precisely with each \TMAGprojection{}, which occurs with each \TMallocation{}.
After a \TMAGprojection{}, \TMmutableExecute{} grows only by the allocated \TMobj{} exactly when the allocator is in the previous mutable set, otherwise it is unchanged.
That is, in the case of \COQallocate{}, operational mutability must only grow by precisely the allocated object when the allocator is \TMmutable{}.

These definitions rely upon the \COQobjExisted{} judgment that identifies those \TMobjs{} either \TMalive{} or \TMdead{} in a \TMsystemState{}.
\xmakefirstuc{\TMobjs} that existed must be kept disjoint from \TMobjs{} that are valid in a \TMAGprojection{}, as \TMAGprojecting{} to a previously existing \TMobj{} violates existing information flows.
In theorems not involving a \TMsystemState{}, the set of \TMobjs{} that existed is a free variable and \TMAGprojections{} are stated targeting references outside this set.
Although this pattern obliges the reader to remember hypotheses not present in the theorem definition, the theorems are more general and easier to apply.

\begin{figure}
\COQDOCmutablePotAccExecuteApprox % - potential operational mutability restricted to initially existing objects is bounded by initial potential mutability.
\COQDOCmutableApproxMutated % - mutated restricted to initially existing objects is bounded by initial potential mutability.
\caption{Initial \TMpotAccMutability{} and Initial \TMmutability{} are upper bounds on \TMmutation{} for existing \TMobjs{}. \label{fig:flow:mutableApproxMutated}}
\end{figure}

Both potential \TMmutability{}, and \TMmutablePotAccExecute{} only place an upper bound on the information flow between existing \TMobjs{}, and cannot describe what information flows will come to exist to new \TMobjs{}.
\xmakefirstuc{\TMmutablePotAccExecute} can only grow by the allocated \TMobj{}, such with the case of all \TMpotAcc{} approximating \TMops{}.
Therefore, the \TMmutablePotAccExecute{} between new \TMobjs{} cannot escalate the existing \TMmutablePotAccExecute{}, when restricted to existing \TMobjs{}, because these sets are disjoint.
At each step, the \TMmutablePotAccExecute{} restricted to the existing \TMobjs{} must remain constant.
In the base case, the \TMmutablePotAccExecute{} is identical to the potential \TMmutability{}.
Finally, by transitivity, what is initially potentially \TMmutable{} is an upper bound on all \TMmutation{} between existing \TMobjs{}.

This result enables analysis of a \TMsystemState{} to statically determine an upper bound on the future \TMmutability{} of all \TMsubsystems{} present by simply examining the initial potential \TMmutability{}.
Not only is \TMpotAcc{} relevant to solving the safety problem, but it also directly approximates potential \TMmutation{} through the \COQmutable{} judgment.
Both definitions are computable and decidable leading to a fully computable and decidable result that can be applied to reason about security policies.

The ability to successfully implement a security policy is dependent on the ability to model how changes in behavior impact the potential for information flow within a system.
The definition of \COQmutable{} is simple and follows directly from permission-based reasoning, yet its results form a persistent upper-bound on mutation through the life of the system.
By examining how restricting capabilities can restrict potential mutability, it is possible to model the behavior of application-based security policies such as confinement.
