\tikzset{lcap/.style={->,semithick}}
\tikzset{bicap/.style={<->,semithick}}
\tikzset{acc/.style={minimum size=8mm,node distance=22mm}}

%% Here are names of theorems and references that will be used throughout this section.
%% Scope them so they are not visible elsewhere

\begingroup

\newcommand{\modellbl}[0]{SW Model}
\newcommand{\prlemma}[0]{Principle Lemma}
\newcommand{\thmauthred}[0]{Lemma 4}
\newcommand{\thmMutationIsMutable}[0]{Lemma 5}
\newcommand{\thmmain}[0]{Theorem 1}
\newcommand{\defmutable}[0]{Definition 1}

%% We are moving all SW-related commands here.
%% We might get some that are not used, but we're not concerned.

\newcommand{\forallspaced}[2]{\ensuremath{\forall #1 \;,\; #2}}
\newcommand{\existsspaced}[2]{\ensuremath{\exists #1 \;,\; #2}}


%% commands for mathematical concepts

\newcommand{\limeq}[0]{\ensuremath{\preccurlyeq}}
\newcommand{\opmath}[1]{\ensuremath{\xrightarrow {#1}}}
\newcommand{\singleop}[3]{\ensuremath{#1 \opmath{#3} #2}}
\newcommand{\singleopfree}[2]{\ensuremath{#1 \rightarrow #2}}
\newcommand{\multiop}[2]{\ensuremath{#1_0 \opmath{\alpha_1} #1_1 \dots \opmath{\alpha_{#2}} #1_{#2}}}
\newcommand{\multiopfree}[2]{\ensuremath{#1_0 \rightarrow #1_1 \dots \rightarrow #1_{#2}}}

%% commands for mathematical syntax

\newcommand{\mutable}[1]{\ensuremath{\mathbf{mutable}_{#1}}}
\newcommand{\mutableapp}[2]{\ensuremath{\mutable{#1}(#2)}}

\newcommand{\readable}[1]{\ensuremath{\mathbf{readable}_{#1}}}
\newcommand{\readableapp}[2]{\ensuremath{\readable{#1}(#2)}}

\newcommand{\mutated}[1]{\ensuremath{\mathbf{mutated}_{#1}}}
\newcommand{\mutatedapp}[2]{\ensuremath{\mutated{#1}(#2)}}

\newcommand{\potacc}[1]{\ensuremath{\mathbf{PotAcc}_{#1}}}
\newcommand{\potaccapp}[3]{\ensuremath{\potacc{#1}(#2,#3)}}

\newcommand{\diracc}[1]{\ensuremath{\mathbf{DirAcc}_{#1}}}
\newcommand{\diraccapp}[3]{\ensuremath{\diracc{#1}(#2,#3)}}

\newcommand{\daset}[0]{\ensuremath{\mathbf{DASet}}}
\newcommand{\target}[1]{\ensuremath{\mathbf{target}(#1)}}
\newcommand{\rights}[1]{\ensuremath{\mathbf{rights}(#1)}}

\newcommand{\stateset}[0]{\ensuremath{\mathcal{S}}}

\newcommand{\objectset}[0]{\ensuremath{\mathbf{Object}}}

\newcommand{\existset}[1]{\ensuremath{{#1}^{\mathbf{exist}}}}
\newcommand{\deadset}[1]{\ensuremath{{#1}^{\mathbf{dead}}}}
\newcommand{\capset}[1]{\ensuremath{{#1}^{\mathbf{caps}}}}
\newcommand{\existedset}[1]{\ensuremath{{#1}^{\mathbf{existed}}}}

\newcommand{\capsetapp}[2]{\ensuremath{\capset{#1}(#2)}}
\newcommand{\obcap}[0]{\ensuremath{\mathbf{ObCap}}}
\newcommand{\obcapapp}[2]{\ensuremath{\obcap(#1,#2)}}

\newcommand{\rightsset}[0]{\ensuremath{\mathcal{R}}}

\newcommand{\execright}[0]{\ensuremath{\mathbf{exec}}}
\newcommand{\readright}[0]{\ensuremath{\mathbf{rd}}}
\newcommand{\writeright}[0]{\ensuremath{\mathbf{wr}}}
\newcommand{\weakright}[0]{\ensuremath{\mathbf{wk}}}
\newcommand{\allrights}[0]{\ensuremath{\top}}
\newcommand{\norights}[0]{\ensuremath{\emptyset}}

\newcommand{\rightscpo}[0]{\ensuremath{\rightsset_{\bot}}}
\newcommand{\rightsleq}[0]{\ensuremath{\leq}}
\newcommand{\rightstop}[0]{\ensuremath{\top}}
\newcommand{\rightsbot}[0]{\ensuremath{\bot}}

\newcommand{\lub}[0]{\ensuremath{\mathbf{lub}}}

\newcommand{\combine}[0]{\ensuremath{\mathbf{combine}}}
\newcommand{\combineA}[1]{\ensuremath{\combine(#1)}}
\newcommand{\combineB}[3]{\ensuremath{\combineA{#1}(#2,#3)}}

\newcommand{\transaccess}[0]{\ensuremath{\mathbf{transAccess}}}
\newcommand{\transaccessapp}[3]{\ensuremath{\transaccess(#1,#2,#3)}}

\newcommand{\invoke}[0]{\ensuremath{\mathbf{invoke}}}

\newcommand{\create}[0]{\ensuremath{\mathbf{create}}}
\newcommand{\createapp}[2]{\ensuremath{\create(#1,#2)}}

\newcommand{\readfrom}[0]{\ensuremath{\mathbf{readfrom}}}
\newcommand{\readfromapp}[1]{\ensuremath{\readfrom(#1)}}

\newcommand{\wroteto}[0]{\ensuremath{\mathbf{wroteto}}}
\newcommand{\wrotetoapp}[1]{\ensuremath{\wroteto(#1)}}


\newcommand{\ProjPotAcc}[0]{\ensuremath{\mathbf{ProjPotAcc}}}
\newcommand{\ProjPotAccA}[1]{\ensuremath{\ProjPotAcc(#1)}}
\newcommand{\ProjPotAccB}[3]{\ensuremath{\ProjPotAccA{#1}(#2,#3)}}
\newcommand{\ProjPotAccC}[5]{\ensuremath{\ProjPotAccB{#1}{#2}{#3}(#4,#5)}}

\newcommand{\AccessRel}[1]{\ensuremath{\mathcal{#1}}}
\newcommand{\AccessRelApp}[3]{\ensuremath{\AccessRel{#1}(#2,#3)}}

\newcommand{\AccessSeq}[1]{\ensuremath{\mathcal{#1}}}
\newcommand{\AccessSeqElem}[2]{\ensuremath{\AccessSeq{#1}_{#2}}}

\newcommand{\mutableLineage}[1]{\ensuremath{\mathbf{mutableLin}_{#1}}}
\newcommand{\mutableLineageApp}[2]{\ensuremath{\mutableLineage{#1}(#2)}}

\newcommand{\lineage}[1]{\ensuremath{\mathbf{Lin}_{#1}}}
\newcommand{\linleq}[0]{\ensuremath{\limeq}}


\newcommand{\maximal}[0]{\ensuremath{\mathbf{maximal}}}


\chapter{The SW Model}
\label{ch:SW}

\TMmodelName{} was initially conceived as a mechanical verification of the \TMSW{} confinement proof \cite{ShapiroWeber2000} with some additional features.
During the verification process, \TMmodelName{} diverged from \TMSW{} in response to extensions, verification pressures, and discovered flaws.
As \TMSW{} informs much of \TMmodelName{}, this chapter presents a comparison of the two models and proposing corrections to \TMSW{} based on the \TMmodelName{} proof.

This chapter begins by discussing some of the differences between \TMSW{} and \TMmodelName{} and proposes some minor changes that are not crucial to the heart of the proof, but necessary for completeness.
It then presents the two major flaws of \TMSW{} identified by \TMmodelName{}: the base case of the main theorem and Lemma 4.
It concludes with a proposal that fits the verified solution of \TMmodelName{} into \TMSW{} to form a result that does not significantly alter the main theorem of \TMSW{}.

\section{Differences Between Models}
\label{sec:SW:model}
\label{sec:SW:model:def}

Although \TMmodelName{} and \TMSW{} are very similar, they differ in a few critical ways.
Their system states differ as \TMSW{} represents the state of \TMobjs{} in disjoint sets instead of tagging each \TMobj{} with meta-data in the map.
The sets are \existset{S} and \deadset{S} with all remaining objects in \(S\) as being available, and correspond to \COQalive{} and \COQunborn{} \TMobjLabels{}.
The set of objects \objectset{} in \TMSW{} is permitted to be infinite where it is presently finite in \TMmodelName{}, though this is addressed in \Cref{sect:future:objReclamation}.
Additionally, \TMmodelName{} contains more \TMobj{} meta-data and uses an \TMidx{} structure in \TMobjs{} to refer to \TMcaps{}.

This chapter uses different font faces to represent concepts in each proof.
Definitions from \TMSW{} are written in \ensuremath{\mathbf{bold face}}, while definitions from \TMmodelName{} are written in \ensuremath{regular face}.
Unless otherwise specified, most definitions herein are referring to \TMSW{} definitions or updated definitions from this chapter.

\begin{figure}
  \begin{tabular}{l}
    If \( \singleop{S_n}{S_{n+1}}{\createapp{p}{a}} \) and \( a \subseteq \capsetapp{S_n}{p} \), then \\
    \( \textnormal{let} o' \in \objectset - \existset{S_n} - \deadset{S_n} \; \textnormal{in} \)
  \end{tabular}
  \[
  \begin{array}{rl}
    \existset{S_{n+1}} & = \existset{S_n} \cup \{o'\} \\
    \capset{S_{n+1}} & = \existset{S_n}[o' \rightarrow a][p \rightarrow \capsetapp{S_n}{p} \cup \obcapapp{o'}{\rightsset}] 
  \end{array}
  \]
  \caption{\modellbl{} : \create{} operation. \label{def:SW:create}}
\end{figure}

Most of the operations in \TMmodelName{} are identical to those in \TMSW{} with the exception that they are described using \TMidxs{} instead of \TMcaps{}.
The \create{} operation has been renamed to \COQallocate{}, the \invoke{} operation has been renamed \COQsend{}, and the \execright{} access right has been renamed to \COQtx{} to avoid confusion in terminology.
Additionally, the \COQsend{} operation includes an optional reply \TMcap{}.
The \TMSW{} \create{} operation will receive special attention in this chapter.

\begin{figure}
Let \rightscpo be the CPO where \( \rightsbot = \norights, \rightstop = \rightsset, \rightscpo = 2^{\rightsset} \text{ and } \rightsleq\) is defined by 
\[ x \subseteq y \Rightarrow x \rightsleq y \; \forallspaced{x,y}{\in \rightsset} \]
\caption{\modellbl{} : \rightscpo definition.\label{def:SW:rights-cpo}}
\end{figure}

\TMmodelName{} does not define a complete partial order over access rights, but instead lifts this CPO to \TMaccessGraphs{} by subset inclusion and \COQtransfer{}.
All of the relationships between access rights are captured in the \COQtransfer{} relation, similar to \transaccess{} in \TMSW{}
The definition of \rightscpo{} from \TMSW{} is included in \Cref{def:SW:rights-cpo} for reference.

\begin{figure}
If \ensuremath{S \in \stateset}, then we define \ensuremath{\diracc{S} : \objectset \times \objectset \rightarrow \rightscpo} by
\[ \diraccapp{S}{x}{y} = \mathbf{lub}(\{ a | (x,y,a) \in \mathbf{DASet}\}) \]
where
\[
  \begin{array}{rll}
    \multicolumn{3}{l}{\daset =} \\
    & \{(o,\target{c},\rights{c}) & | o \in \objectset, c \in \capsetapp{S}{o}\} \\
    \cup & \{(\target{c},o, \allrights) & | o \in \objectset, c \in \capsetapp{S}{o}, \execright \in \rights{c} \} 
  \end{array}
\]
\caption{\label{def:SW:DirAcc} \modellbl{} : \diracc{} definition.}
\end{figure}

The definition of \COQdirAcc{} in \TMmodelName{} closely follows the definition of \diracc{} in \TMSW{}.
The primary difference is that the \execright{} case does not produce \(\top\) authority in \COQdirAcc{}.
In \TMmodelName{} this is also managed as emergent behavior in the \COQtransfer{}.

\begin{figure}
If \( S \in \stateset \), then the potential access relation, \( \potacc{S} \) is the limit of the series \( T_0,T_1,T_2,\dots \) where
\begin{align*}
T_0 &= \diracc{S} \\
\forallspaced{x,y}{T_{i+1}(x,y)} &= \lub\{T_i(x,y),\combineB{T_i}{x}{y}\})
\end{align*}
where 
\[ \combineB{A}{x}{z} = \lub(\{a | \exists y \in \objectset \; \textnormal{such that} \; a = \transaccessapp{x}{y}{z}\}) \]
and
\[ 
\transaccessapp{x}{y}{z} = 
\left\{
\begin{array}{ll}
  \bot & \textnormal{if} \; A(x,y) = \bot \\
  A(y,z) & \textnormal{if} \; \{\readright, \execright\} \cap A(x,y) \neq \norights \} \\
  \{\weakright\} & \textnormal{if} \; \{\weakright, \readright, \execright\} \cap A(x,y) = \{\weakright\} \land \\
                 & \{\weakright,\readright\} \cap A(y,z) \neq \norights \\
  \rightsbot & \textnormal{otherwise} \\
\end{array}
\right.
\]
\caption{\modellbl{} : \potacc{} definition. \label{def:SW:PotAcc}}
\end{figure}

Most of the differences between \TMmodelName{} and \TMSW{} appear in how they structure \TMpotAcc{}.
In \TMSW{}, \potacc{} is defined directly in terms of \diracc{} as the closure of \combine{}: a very large combinator of the least upper bound of \transaccess{}.
\TMmodelName{} breaks these relations apart using \TMaccessGraphs{} as an intermediate structure.
Therefore, the definition of \COQpotAcc{} is meaningful over many different \TMaccessGraphs{}, not only the \TMdirAccAG{}.
This allows \TMmodelName{} to restructure \transaccess{} into \COQtransfer{} as a micro-operation of access right transfer.
This eases mechanical reasoning about \TMpotAcc{} from any reachable \TMaccessGraph{} by allowing justifications in the closure to be simply reordered.
Therefore, \COQtransfer{} subsumes the previously mentioned properties from \diracc{} and \rightscpo{}.

Reflexivity in \TMpotAcc{} is another major difference between \TMSW{} and \TMmodelName{}.
\TMSW{} does not define \diracc{} or \potacc{} as reflexive closures, only transitive ones.
\TMmodelName{} ensures that \COQpotAcc{} contains all reflexive \TMaccessEdges{} through \COQtransfer{} causing \TMdirAcc{} to remain a simple translation.
Reflexivity of access will become very important in the discussion below.

\section{Minor Errata}
\label{sec:SW:errata}

\begin{figure}
If \ensuremath{S \in \stateset}, then
\[ \mutableapp{S}{E} = \{y|\exists x \in E, \{\writeright,\execright\} \cap \potaccapp{S}{x}{y} \neq \norights \} \]
\caption{\modellbl{} : \mutable{}. \label{def:SW:mutable}}
\end{figure}

\begin{figure}
If \ensuremath{S \in \stateset}, then
\[ \mutableapp{S}{E} = 
\begin{array}{ll}
  \{y | \exists x \in E, & \{\writeright, \execright\} \cap \potaccapp{S}{x}{y} \neq \norights \lor \\
  & \{\readright, \weakright\} \cap \potaccapp{S}{y}{x} \neq \norights  \} 
\end{array}
\]   
\caption{Correction to \mutable{}. \label{def:SW:errata:mutable}}
\end{figure}

There are a few minor errata in \TMSW{} that, while their meaning may be inferred given context, must be corrected before addressing large changes.
The four alterations are to \mutable{} (resp. \readable{}), \rightsset{}, \transaccess{}, and information flow via the \create{} operation.
\Cref{def:SW:mutable} presents \mutable{} from \TMSW{}, which is intended to capture all potential information flow.
However, the \readright{} and \weakright{} rights are omitted and \readable{} is simply the inverse of \mutable{}.
Because \transaccess{} in \potacc{} propagates existing permissions, \readright{} and \weakright{} access rights will never be considered.
This is fixed by adding the missing condition to \mutable{} using the definition in \Cref{def:SW:errata:mutable}.
It is not necessary to alter \readable{}, as the change to \mutable{} corrects both definitions.

\begin{figure}
Let \rightscpo{} be the CPO where \( \rightsbot = \norights, \rightstop = \rightsset, \rightscpo = 2^{\rightsset} \text{ and } \rightsleq\) is defined by 
\[
\forallspaced{x,y \in \rightsset}{
x \rightsleq{} y =
\left\{
\begin{array}{ll}
x - \{\weakright,\readright\} \subseteq y - \{\weakright,\readright\} & \text{if } \weakright \in x \land \readright \in y\\
x \subseteq y & otherwise
\end{array}
\right.}
\]
\caption{Correction to \modellbl{} : \rightscpo.\label{def:SW:errata:rights-cpo}}
\end{figure}

The \weakright{} permission is interpreted inconsistently in \TMSW{} and is not always considered a positive permission.
The \readright{} permission conveys all of the authority that the \weakright{} permission does.
Therefore, \readright{} should be considered a proper subtype of \weakright{}.
In practice, these two permissions do not appear together, however, the model does not exclude this possibility.
By implementing this change in \rightscpo{}, it will trickle down to all other aspects of the proof including \lub{} and \potacc{} definitions.


\begin{figure}
\[ 
\transaccessapp{x}{y}{z} = 
\left\{
\begin{array}{ll}
  A(y,z) & \textnormal{if} \; \{\readright, \execright\} \cap A(x,y) \neq \norights \} \\
  A(y,z) & \textnormal{if} \; \{\writeright, \execright\} \cap A(y,x) \neq \norights \} \\
  \{\weakright\} & \textnormal{if} \; \{\weakright, \readright, \execright\} \cap A(x,y) = \{\weakright\} \land \\
  & \{\weakright,\readright\} \cap A(y,z) \neq \norights \\
  \rightsbot & \textnormal{otherwise} \\
\end{array}
\right.
\]
\caption{Correction to \transaccess.\label{def:SW:errata:transaccess}}
\end{figure}


In \TMSW{}, the underlying transitive operation capturing potential permission transfers is defined by \transaccess{}.
Unfortunately, \transaccess{} neglects a discussion of \writeright{} permissions being capable of transferring other permissions.
The missing case should invert the transitivity rule handling the \readright{} and \execright{} case.
The updated definition also omits first \(\bot\) case, as it is fully subsumed by the last case.

The last minor correction involves the definition of information flow for the \create{} operation.
The \readfrom{} and \wroteto{} relations are used to define potential information flow for a given operation.
It is important to note that the \create{} operation may send capabilities from the parent to the child.
\emph{Capability flow is information flow} and it must be captured to avoid incorrect analysis.
The only correction to the \TMSW{} model is to alter the definition of \wrotetoapp{\createapp{p}{a}} to be \(\{p,n\}\), where \(n\) is the newly created object.

\section{Major Concerns}

This section addresses two of the major issues in the \TMSW{} verification.
The first demonstrates that the main theorem's induction hypothesis is invalid in the base case and proposes a simple solution.
The second illustrates a flaw in \thmauthred{}, required for the main theorem.

It is worth noting that the discovery of both flaws occurred as the direct outcome of interacting with the proof assistant.
Attempts to verify similar theorems in \TMmodelName{} arrived at obviously erroneous goals which, with some careful examination, lead to the counterexamples presented below.
Ultimately, these counterexamples informed changes in the structure of the proof execution leading to a nearly identical conclusion.

\subsection{Invalid Induction Hypothesis}
\label{sec:SW:basecase}
\begin{figure}
\[
\mutatedapp{E}{\multiop{S}{n}} \cap \existedset{S_0} \subseteq \mutableapp{S_0}{E}
\]
\caption{\modellbl{} : Main Theorem.\label{def:SW:mainTheorem}}
\end{figure}

\begin{figure}
Assume that for all sets \(F\),
\[
\mutatedapp{F}{\multiop{S}{n-1}} \cap \existedset{S_0} \subseteq \mutableapp{S_0}{F}
\]
\caption{\modellbl{} : Induction Hypothesis.\label{def:SW:mainIH}}
\end{figure}

\begin{figure}
\[
\forallspaced{E}{E \subseteq \mutableapp{S_0}{E}}
\]
\caption{Property: \mutable{} is non-decreasing.\label{prop:SW:mutable:nondecreasing}}
\end{figure}


The main theorem \TMSW{} attempts to prove the theorem in \Cref{def:SW:mainTheorem} by induction on \(n\).
This relies on the induction hypothesis for n-1 as shown in \Cref{def:SW:mainIH}.
\TMSW{} claims that the base case is trivial, yet it is straightforward to violate.
\mutated{} is always growing the initial subsystem \(E\), while \mutable{} only considers what \(E\) can mutate via \potacc{}.
When no operations are performed, the base case reduces to the property of \Cref{prop:SW:mutable:nondecreasing}: \mutable{} is non-decreasing.
However, \thmMutationIsMutable{} of the \TMSW{} excludes the initial subsystem \(E\) from what is considered mutable, indicating that the base case does not hold.

To help visualize counterexamples, concrete instances are described visually.
System states are represented as graphs in the same manner as \TMmodelName{}.
As \TMSW{} considers all \TMobjs{} \TMactive{}, objects are circles and capabilities are edges in the \TMsystemState{} diagram.  
As with \TMmodelName{} diagrams, capabilities are labeled with their access right set and the border and fill of objects indicate their label.
{\bf Dead} objects are gray, an object that {\bf exists} has a solid border, and an {\bf available} object has a dashed border.
Access relations are also visually represented, but they consist only of relationships in \rightscpo{} between \TMobjs{} which have no border.


\begin{figure}
\centering
\begin{tikzpicture}[auto]
  
  \matrix[column sep=10mm, row sep=10mm, ampersand replacement=\&]{

  \node [subjectAlive]           (A)    {A}; \&
  \node [subjectAlive]           (A')   {B}; \&

  \node [acc]                    (dA)   {A}; \&
  \node [acc]                    (dA')  {B}; \\ 
  };
  \draw [capArrow] (A) edge  node[midway,above] {\tiny \(\rightsset\)} (A');
  \draw [capArrow] (dA) edge  node[midway,above] {\tiny \(\allrights\)} (dA');

\end{tikzpicture}
\[
\mathbf{mutated}_{\{A\}}(S_0) \cap \{A,B\} = \{A\} \nsubseteq \{B\} = \mathbf{mutable}_{S_0}(\{A\})
\]
\caption{Contradiction: \label{fig:SW:mainIH}System State \(S_0\) and \potacc{S_0} which violates the induction hypothesis.\label{cont:SW:mainIH}}
\end{figure}

The counterexample in \Cref{cont:SW:mainIH} illustrates that the present definition of \mutable{} is not non-decreasing.
When no operations are performed, \(\mutable{\ensuremath{E}} = E\).
However, \daset{} does not capture this access, and therefore it must not be present in \diracc{\ensuremath{S_0}} or \potacc{\ensuremath{S_0}}.
Therefore, \(\mutableapp{\ensuremath{S_0}}{\{A\}} = \{B\}\) which is not a superset of \(\{A\}\).

\begin{figure}
  \[
  \begin{array}{rll}
    \multicolumn{3}{l}{\daset =} \\
    & \{(o,\target{c},\rights{c}) & | o \in \objectset, c \in \capsetapp{S}{o}\} \\
    \cup & \{(\target{c},o, \allrights) & | o \in \objectset, c \in \capsetapp{S}{o}, \execright \in \rights{c} \} \\
    \cup & \{(o,o, \allrights) & | o \in \objectset \} \\  
  \end{array}
  \]
  \caption{Correction to \daset{}. \label{def:SW:fix:DASet}}
\end{figure}


This problem can be avoided by altering the definition of \mutable{} to conform to the property in \Cref{prop:SW:mutable:nondecreasing}.
There are two potential solutions: alter \mutable{} to have this property directly, or alter the definition of \daset{} to cause \potacc{} to be a transitive reflexive closure.
In \TMmodelName{}, \mutable{} was altered to be non-decreasing in addition to altering \potacc{} to be a transitive reflexive closure without altering \daset{}.
In hindsight, it is better to alter \diracc{} to be reflexive, as in \Cref{def:SW:fix:DASet}, as this simplifies relations between access relations.
For more information on this future work in \TMmodelName{}, see \Cref{sect:future:complexities}.

\subsection{Violating \thmauthred{}}
\label{sec:SW:authred}

\begin{figure}
If \( \singleop{S_0}{S_1}{\alpha} \), then for all E,
\[ \mutableapp{S_1}{E} \cap \existedset{S_0} \subseteq \mutableapp{S_0}{E \cap \existedset{S_0}} \]
\caption{\modellbl{} : \thmauthred{} \label{def:SW:auth_red}}
\end{figure}

The main theorem also relies on \thmauthred{}, which also contains flaws that must be corrected to present a solution.
\thmauthred{} in the \modellbl{} misquantifies the initial subsystem \(E\) and is consequentially too general to be correct.
The subsystem \(E\) admits \emph{any} subset of \objectset, including any \emph{unborn} objects.
This permits unborn objects that will be descended from subsystems \emph{outside} of \(E \cap \existedset{S}\).
Because \(E\) is unrestricted in this way, it is possible to construct an example where elements of some initial \(E\) become children of an external subsystem, connecting \(E\) to a previously unconnected subsystem.

\begin{figure}
  \centering
  \begin{tikzpicture}[auto]
    
    \matrix[column sep=10mm, row sep=10mm, ampersand replacement=\&]{
      
      \node[subjectAlive]     (A)  {\(A_1\)}; \&
      \&
      \node[subjectAlive]     (B)  {\(B_1\)}; \&[+5mm]
      \node[acc]     (dA')   {\(A_1\)}; \&
      \&
      \node[acc]     (dB)    {\(B_1\)};
      \\
      \node[subjectAlive]     (A') {\(A_2\)}; \&
      \node[subjectUnborn]    (U)  {\(A_u\)}; \&
      \node[subjectAlive]     (B') {\(B_2\)}; \& 
      \node[acc]     (dA)    {\(A_2\)}; \&
      \node[acc]     (dU)    {\(A_u\)}; \&
      \node[acc]     (dB')   {\(B_2\)};
      \\
    };
    
    
    \path [capArrow] (A) edge [bend left=15]  node[midway,right] {\tiny \rightsset} (A');
    \path [capArrow] (A') edge [bend left=15]  node[midway,left] {\tiny \rightsset} (A);
    
    \path [capArrow] (B) edge [bend left=15]  node[midway,right] {\tiny \rightsset} (B');
    \path [capArrow] (B') edge [bend left=15]  node[midway,left] {\tiny \rightsset} (B);

    \path [capArrow] (dA) edge [bend left=15] node[midway,left] {\tiny \allrights} (dA');
    \path [capArrow] (dA') edge [bend left=15] node[midway,right] {\tiny \allrights} (dA);
    
    \path [capArrow] (dB) edge [bend left=15]  node[midway,right] {\tiny \allrights} (dB');
    \path [capArrow] (dB') edge [bend left=15] node[midway,left] {\tiny \allrights} (dB);
    
    %% Self loops
    
    \path [capArrow] (A) edge [loop left]  node[midway,above] {\tiny \rightsset} (A);
    \path [capArrow] (B) edge [loop left]  node[midway,above] {\tiny \rightsset} (B);
    
    \path [capArrow] (dA) edge [loop left]  node[midway,above] {\tiny \allrights} (dA);
    \path [capArrow] (dA') edge [loop left]  node[midway,above] {\tiny \allrights} (dA');
    \path [capArrow] (dB) edge [loop right]  node[midway,above] {\tiny \allrights} (dB);
    \path [capArrow] (dB') edge [loop right]  node[midway,above] {\tiny \allrights} (dB');
    
  \end{tikzpicture}
  \caption{\label{fig:SW:authred:S0}System State \(S_0\) and \(\mathbf{PotAcc}_{S_0}\) to violate \thmauthred{}.}
\end{figure}

\begin{figure}
  \begin{center}
    \begin{tikzpicture}[auto]
      
      \matrix[column sep=10mm, row sep=10mm, ampersand replacement=\&]{
        
        \node[subjectAlive]     (A)  {\(A_1\)}; \&
        \&
        \node[subjectAlive]     (B)  {\(B_1\)}; \&[+5mm]
        \node[acc]     (dA')   {\(A_1\)}; \&
        \&
        \node[acc]     (dB)    {\(B_1\)};
        \\
        \node[subjectAlive]     (A') {\(A_2\)}; \&
        \node[subjectUnborn]    (U)  {\(A_u\)}; \&
        \node[subjectAlive]     (B') {\(B_2\)}; \& 
        \node[acc]     (dA)    {\(A_2\)}; \&
        \node[acc]     (dU)    {\(A_u\)}; \&
        \node[acc]     (dB')   {\(B_2\)};
        \\
      };

        \path [lcap] (B) edge  node[midway,above,sloped] {\tiny \rightsset} (U);
        \path [lcap] (U) edge  node[midway,below,sloped] {\tiny \rightsset} (B');

        %% come back and change each of these to two labeled caps.

      \path [capArrow] (A) edge [bend left=15]  node[midway,right] {\tiny \rightsset} (A');
      \path [capArrow] (A') edge [bend left=15]  node[midway,left] {\tiny \rightsset} (A);
            
      \path [capArrow] (B) edge [bend left=15]  node[midway,right] {\tiny \rightsset} (B');
      \path [capArrow] (B') edge [bend left=15]  node[midway,left] {\tiny \rightsset} (B);

      \path [capArrow] (dA) edge [bend left=15] node[midway,left] {\tiny \allrights} (dA');
      \path [capArrow] (dA') edge [bend left=15] node[midway,right] {\tiny \allrights} (dA);
      
      \path [capArrow] (dB) edge [bend left=15]  node[midway,right] {\tiny \allrights} (dB');
      \path [capArrow] (dB') edge  node[midway,left] {\tiny \allrights} (dB);
      
      \path [capArrow] (dB) edge  node[midway,below,sloped] {\tiny \allrights} (dU);
      \path [capArrow] (dU) edge [bend left=15]  node[midway,above,sloped] {\tiny \allrights} (dB);

      \path [capArrow] (dU) edge  node[midway,above] {\tiny \allrights} (dB');
      \path [capArrow] (dB') edge [bend left=15]  node[midway,below] {\tiny \allrights} (dU);
        
        %% Self loops
        
        \path [lcap] (A) edge [loop left]  node[midway,above] {\tiny \rightsset} (A);
        \path [lcap] (B) edge [loop left]  node[midway,above] {\tiny \rightsset} (B);
        \path [lcap] (dA) edge [loop left]  node[midway,above] {\tiny \allrights} (dA);
        \path [lcap] (dA') edge [loop left]  node[midway,above] {\tiny \allrights} (dA');
        \path [lcap] (dB) edge [loop right]  node[midway,above] {\tiny \allrights} (dB);
        \path [lcap] (dB') edge [loop right]  node[midway,above] {\tiny \allrights} (dB');
        \path [lcap] (dU) edge [loop left]  node[midway,above] {\tiny \allrights} (dU);
        
        
    \end{tikzpicture}
  \end{center}
  \caption{\label{fig:SW:authred:S1}System State \(S_1\) and \(\mathbf{PotAcc}_{S_1}\) to violate \thmauthred{}.}
\end{figure}

\pagebreak[4]

Consider the result of a \create{} operation on \(S_0\) as in \Cref{fig:SW:authred:S0}.
If the operation is:
\[\singleop{S_0}{S_1}{\createapp{p}{\{\obcapapp{\ensuremath{B_2}}{\rightsset}\}}}\] 
then assuming that there exists an object \(A_u \in \objectset \), the result of this \create{} operation will be \(S_1\), shown in \Cref{fig:SW:authred:S1}.
This example is constructed such that the error occurs using either the original \TMSW{} definitions or the modifications to this point.
With \(A_u\) as the unborn object to be created by \ensuremath{B_2}, examine \thmauthred{} with \(E = \{A_1, A_2, A_u\}\).
It is now possible for \(\{A_1, A_2, A_u\}\) to modify \(B_*\) where before it was not.

\begin{figure}
\begin{align*}
\mutableapp{S_0}{\{A_1,A_2,A_u\} \cap \existedset{S_0}} &= \mutableapp{S_0}{\{A_1,A_2,A_u\}} \cap \{A_1,A_2,B_1,B_2\} \\
&= \{A_1,A_2,A_u\} \cap \{A_1,A_2,B_1,B_2\} \\
&= \{A_1,A_2\} \\
\\
\mutableapp{S_1}{\{A_1,A_2,A_u\}} \cap \existedset{S_0} &= \mutableapp{S_1}{\{A_1,A_2,A_u\}} \cap \{A_1,A_2,B_1,B_2\} \\
&= \{A_1,A_2,A_u,B_1,B_2\} \cap \{A_1,A_2,B_1,B_2\} \\
&= \{A_1,A_2,B_1,B_2\}
\end{align*}
\caption{Value of \mutable{} for \(S_0\) and \(S_1\).}
\end{figure}

\begin{figure}
\thmauthred{} states that
If \( \singleop{S_0}{S_1}{\alpha} \), then for all E,
\[
\mutableapp{S_1}{E} \cap \existedset{S_0} \subseteq \mutableapp{S_0}{E \cap \existedset{S_0}}
\]
But for the example defined above, when \(E = \{A_1, A_2, A_u\}\) :
\[
\mutableapp{S_1}{E} \cap \existedset{S_0} = \{A_1,A_2,B_1,B_2\} \nsubseteq \{A_1,A_2\} = \mutableapp{S_0}{E \cap \existedset{S_0}}
\]
\caption{Contradiction of \thmauthred{}. \label{cont:SW:authred}}
\end{figure}

As this example uses \allrights{} relationships, no alterations thus far could not prevent this problem.
The reflexivity alteration for \daset{} is subsumed by including sufficient self-authority to cause reflexivity.
Corrections in \rightscpo{} and \mutable{} are uninteresting as all capabilities are fully permissive.
The contradiction does not examine the definition of \mutated{} and therefore does not involve the modification of \wroteto{} from \Cref{sec:SW:basecase}.

\section{Solution}
\label{sec:SW:solution}

The absurdity of \thmauthred{} arises from the ability to ask questions about unborn objects without knowing their lineage.
The main theorem never quantifies \thmauthred{} with such an absurd E, but the analysis skips this detail.
The solution is similar to \TMmutableExecute{} from \TMmodelName{}.

\subsection{Relating Access Relations}

\begin{figure}
  \begin{tabular}{l}
    If \(\AccessRel{A} : \objectset \times \objectset \rightarrow\ \rightscpo \) is a resource relationship, \\
    then \(\ProjPotAccA{\mathcal{A}} : \objectset \times \objectset \rightarrow \AccessRel{A}\) is defined to be:
  \end{tabular}
\[
\ProjPotAccC{\AccessRel{A}}{p}{u}{x}{y} = 
\left\{
\begin{array}{ll}
  \top & \textnormal{if} \; x = p \land y = u \\
  \top & \textnormal{if} \; x = u \land y = p \\
  \top & \textnormal{if} \; x = p \land y = p \\  
  \top & \textnormal{if} \; x = u \land y = u \\
  \AccessRelApp{A}{p}{y} & \textnormal{if } x = u \\
  \AccessRelApp{A}{x}{p} & \textnormal{if } y = u \\
  \AccessRelApp{A}{x}{y} & \textnormal{otherwise}
\end{array}
\right.
\]
\caption{Definition of \ProjPotAcc{}.\label{def:SW:ProjPotAcc}}
\end{figure}



Integrating solutions from \TMmodelName{} begins by describing the relationships between access relations over each operation.
From the updated definitions, \potacc{} is a transitive reflexive closure and should approximate all operations except \create{}.
It is impossible for \potacc{} to approximate \create{}, as there is no advance knowledge about the lineage of new objects.
However, in the worst case, a newly created object is just as permissive as its creator.
This relationship between access relations is given by \ProjPotAcc{} in \Cref{def:SW:ProjPotAcc}.
Given \ProjPotAcc{}, it is possible to demonstrate that \potacc{} approximates all operations, as shown in \Cref{thm:SW:ProjPotAccOp}.

\begin{figure}
If \( \singleop{S_0}{S_1}{\createapp{p}{a}} \) then if \(u\) is selected as a newly created object,
\[
\potacc{S_1} \leq \ProjPotAccB{\potacc{S_0}}{p}{u}
\]
and in all other cases
\[
\potacc{S_1} \leq \potacc{S_0}
\]
where the definition of \(\leq\) is overloaded on access relations to have the meaning
\[
\AccessRel{A} \leq \AccessRel{B} \Leftrightarrow \forall x,y \; \AccessRelApp{A}{x}{y} \leq \AccessRelApp{B}{x}{y}
\]
\caption{Theorem: \ProjPotAcc{} approximates all operations.\label{thm:SW:ProjPotAccOp}}
\end{figure}


%% This proof is solved by very exhaustive case analysis that follows the \COQagProjectEndow{} theorem in \Cref{fig:safety:AGproject} in \TMmodelName{}.
This proof is solved by very exhaustive case analysis similar to the \TMmodelName{} theorem \COQagProjectEndow{} in \Cref{fig:safety:AGproject}.
It should also be apparent that under a transitive reflexive closure, if \AccessRel{A} has converged over \combine{}, then \ProjPotAccA{\AccessRel{A}} has also converged.
This proof is less obvious, but is similar to the \TMmodelName{} theorem \COQAGProjectMaximal{} in \Cref{fig:flow:AGProjectMaximal}.
Such converged resource relationships will be called \maximal{} throughout the remainder of this chapter.

\begin{figure}
If \multiop{S}{n} is an execution, then \multiop{\AccessSeq{P}}{n} is defined such that
\[
\forall i \; \AccessSeqElem{P}{i} = \potacc{S_i}
\]
\caption{Definition of potential access relation sequences.\label{def:SW:potAccRel}}
\end{figure}

\begin{figure}
  \begin{tabular}{l}
    If \multiop{\AccessSeq{P}}{n}\\
    and \multiop{\AccessSeq{Q}}{n} are \maximal{} access relation sequences,\\
    then \rightsleq is overloaded as
  \end{tabular}
\[
\multiopfree{\AccessSeq{P}}{n} \rightsleq \multiopfree{\AccessSeq{Q}}{n} \Leftrightarrow \forall i \; \AccessSeqElem{P}{i} \rightsleq \AccessSeqElem{Q}{i}
\]
\caption{Definition of one maximal access relation sequence approximating another.\label{def:SW:potAccRelApprox}}
\end{figure}

It is now possible to define notation relating sequences of access relations that are sourced from executions, as shown in \Cref{def:SW:potAccRel,def:SW:potAccRelApprox}.
For clarity, the \rightsleq{} operator is further overloaded for \maximal{} relation sequences.

\subsection{Mutable Lineage}

\begin{figure}
  \[ \mutableapp{\AccessRel{A}}{E} =
  \begin{array}{ll}
    \{y | \exists x \in E, & \{\writeright, \execright\} \cap \AccessRelApp{A}{x}{y} \neq \norights \; \lor \\ 
    & \{\readright, \weakright\} \cap \AccessRelApp{A}{y}{x} \neq \norights\}
  \end{array}
  \]
\caption{Parameterized definition of \mutable{}.\label{SW:def:newMutable}}
\end{figure}

With the introduction of transformations between access relationships, the definition of mutability is too specific to incorporate new relationships as it is still dependent on \potacc{}.
Therefore, \mutable{} must be parameterized over any access relationship \(\AccessRel{A}\) as in \Cref{SW:def:newMutable}.
Although this chapter has striven to preserve notation from \TMSW{}, it is necessary to deviate slightly to accomplish this change.

\begin{figure}
If \AccessRel{A} and \AccessRel{B} are access relations and \(E\) and \(F\) are object sets, then
\[ \AccessRel{A} \rightsleq \AccessRel{B} \land E \subseteq F \; \Rightarrow \mutableapp{\AccessRel{A}}{E} \subseteq \mutableapp{\AccessRel{B}}{F} \]
\caption{Theorem: \mutable{} respects ordering. \label{SW:thm:mutableRespectful}}
\end{figure} 

This form of \mutable{} respects the various ordering relations for all parameters.
Therefore, as the \ProjPotAcc{} relationship preserves these orderings, \mutable{} must respect \ProjPotAcc{} as well.
The theorem in \Cref{SW:thm:mutableRespectful} corresponds to the \COQproperMutableSpec{} theorem from \TMmodelName{} show in \Cref{fig:flow:mutableProperties}.

\begin{figure}
  \begin{tabular}{l}
    If \(\multiop{S}{n}\) is an execution, \\
    and \(\multiop{\AccessSeq{P}}{n} \rightsleq \multiopfree{\AccessSeq{Q}}{n}\), \\
    and \(E \subseteq \existedset{S_0}\), and \( q_i = \multiopfree{\AccessSeq{Q}}{i}\) are sub-sequences
  \end{tabular}
  \[
  \begin{array}{rcl}
    \mutableLineageApp{E}{\AccessSeqElem{Q}{0}} & = & \mutableapp{\AccessSeqElem{Q}{0}}{E} \\
    \mutableLineageApp{E}{\singleopfree{\AccessSeqElem{Q}{i-1}}{\AccessSeqElem{Q}{i}}} & = & \mutableapp{\AccessSeqElem{Q}{i}}{E} \\
    \mutableLineageApp{E}{q_n} & = & \mutableLineageApp{\mutableLineageApp{E}{q_{n-1}}}{\singleopfree{\AccessSeqElem{Q}{n-1}}{\AccessSeqElem{Q}{n}}}
  \end{array}
  \]
  \caption{Definition of \mutableLineage{}.\label{def:SW:mutableLineage}}
\end{figure}

\begin{figure}
  \begin{tabular}{l}
    If \(\multiop{S}{n}\) is an execution, \\
    and \(\multiop{\AccessSeq{P}}{n} \rightsleq \multiopfree{\AccessSeq{Q}}{n}\), \\
    and \(E \subseteq \existedset{S_0}\), \\
    and \( e_i = \multiop{S}{i}\) are sub-executions, \\
    and \( q_i = \multiopfree{\AccessSeq{Q}}{i}\) are sub-sequences
  \end{tabular}
  \[
  \mutatedapp{E}{e_n} \subseteq \mutableLineageApp{E}{q_n}
  \]
  \caption{Theorem: \mutated{} is captured by of a \mutableLineage{}.\label{thm:SW:mutatedmutableLinage}}
\end{figure}

The definition of \mutableLineage{} in \Cref{def:SW:mutableLineage} sets up a sequence of \mutable{}s to capture object lineage while approximating \mutated{}, similar to \COQmutableExecute{} in \TMmodelName{} shown in \Cref{fig:flow:mutableExecute}.
The induction strategy for \mutableLineage{} is identical to \mutated{} except that it starts with what is \mutable{} and cycles \mutable{} at each step.
However, as it will become important to abstract over approximations, the definition does not directly perform reasoning on \potacc{S_n}.
Therefore, the definition of \mutableLineage{} is quite abstract.
If all \potacc{} operations are approximated reflexively, the theorem begins look a bit more like \mutable{} and it should become clear that \mutableLineage{} also approximates \mutated{} as shown in \Cref{thm:SW:mutatedmutableLinage}.
In \TMmodelName{}, this theorem corresponds to \COQmutableDirAccExecuteApproxMutated{} in \Cref{fig:flow:mutatedapprox}.

\begin{figure}
  \begin{tabular}{l}
    If \(E\) and \(F\) are object sets, \\
    and \(\multiopfree{\AccessSeq{P}}{n} \leq \multiopfree{\AccessSeq{Q}}{n}\), \\
    and \( p_i = \multiopfree{\AccessSeq{P}}{i}\) are sub-sequences, \\
    and \( q_i = \multiopfree{\AccessSeq{Q}}{i}\) are sub-sequences,
  \end{tabular}
\[ 
E \subseteq F \; \Rightarrow \mutableLineageApp{E}{p_i} \subseteq \mutableLineageApp{F}{q_i}
\]
\caption{Theorem: \mutableLineage{} respects ordering. \label{SW:thm:mutableLineageRespectful}}
\end{figure} 

Since \mutableLineage{} is defined inductively over \mutable{} -- which respects approximations and subset as shown in \Cref{SW:thm:mutableRespectful} -- it too must respect approximations and subset.
This is presented in \Cref{SW:thm:mutableLineageRespectful} and is similar to the \TMmodelName{} theorem \COQproperMutableExecute{} in \Cref{fig:flow:properMutableExecute}.

\subsection{Lineage}

\begin{figure}
  \begin{tabular}{l}
    If \AccessRel{C} is an access relation, \\
    let \lineage{\AccessRel{C}} be the inductively defined set rooted at a \maximal{} \AccessRel{C}: \\
  \end{tabular}
\[
\begin{array}{rl}
  \mathcal{C} \in \lineage{\AccessRel{C}} & \\
  \ProjPotAccB{\AccessRel{A}}{p}{u} \in \lineage{\AccessRel{C}} & \textnormal{if} \;
          \AccessRel{A} \in \lineage{\AccessRel{C}} \land  
          p \in \AccessRel{A} \land
          u \notin \AccessRel{A}
\end{array}
\]
where definition of \( \in \) is overloaded as \( \forallspaced{o,\AccessRel{A}}{o \in \AccessRel{A} \Leftrightarrow \AccessRelApp{A}{o}{o} \neq \rightsbot }\) 

And let \linleq{} be the CPO defined by:
\[
\begin{array}{rcll}
\AccessRel{A} & \linleq & \AccessRel{A} & \\
\AccessRel{A} & \linleq & \ProjPotAccB{\AccessRel{A}}{p}{u} & \textnormal{if} \; p \in \AccessRel{A} \land u \notin \AccessRel{A}
\end{array}
\]
Note, \(\bot = \AccessRel{C}\) and \linleq{} is anti-symmetric by the definition of subset and \ProjPotAcc.
\caption{Definition : \lineage{R}.\label{SW:def:Lin}}
\end{figure}

The partial order \lineage{\AccessRel{R}} using \ProjPotAcc{} as in \Cref{SW:def:Lin} is defined to assist reasoning about \mutableLineage{}.
The purpose of \lineage{\AccessRel{R}} is to fully abstract all possible lineage hierarchies rooted at access relation \AccessRel{R}.
Primarily, this abstracts all series of create operations, as the preconditions of \lineage{\AccessRel{R}} are always satisfied by any sequence of operations due to state transitions.
Additionally, \lineage{\AccessRel{R}} parameterizes theorems without needing to instantiate them through laborious notation.
The notion of \lineage{R} as a partial order is not directly present in \TMmodelName{} as it would require restructuring many previous theorems.

\begin{figure}
If \(\multiop{S}{n}\) is an execution, \(E \subseteq \existedset{S_0}\), then
\[
\existsspaced {\AccessSeq{L} \in \lineage{\potacc{S_0}}} {\mathbf{PotAcc}_{S_n} \rightsleq \AccessSeq{L}}
\]
or 
\[
\existsspaced {\multiopfree{\AccessSeq{L}}{n} \in \lineage{\potacc{S_0}}} {\multiop{\AccessSeq{P}}{n} \rightsleq \multiopfree{\AccessSeq{L}}{n}}
\]
\caption{Theorem: \lineage{} approximates \potacc{}. \label{thm:SW:ProjPotAccApprox}}
\end{figure}

For any sequence of executions, \potacc{S_n} can be approximated by member of \lineage{\potacc{S_0}} using the theorem from \Cref{thm:SW:ProjPotAccOp}.
This is justified for two reasons.  First, all non-create operations are self-approximating.  
Second, because \potacc{S_0} is \maximal{} and \ProjPotAcc{} is \maximal{} given a \maximal{} input, this sequence must be \maximal{} by induction.
This notion is presented in \Cref{thm:SW:ProjPotAccApprox} and loosely corresponds to the \TMmodelName{} theorem \COQmutableExecuteDirAccSubsetPotAcc{} in \Cref{fig:flow:mutableExecuteSpecializations}.

\begin{figure}
If \multiop{S}{n} is an execution, 
and \(\multiopfree{\AccessSeq{L}}{n} \in \lineage{\potacc{S_0}}\)
and \(E \subseteq \existedset{S_0}\), 
and \( l_i = \multiopfree{\AccessSeq{L}}{i}\) are sub-sequences
\[
\mutableLineageApp{E}{l_i} \subseteq \existedset{S_i}
\]
\caption{Theorem : \mutableLineage{} always {\bf existed}.\label{SW:thm:mutableLineageExisted}}
\end{figure}

\mutableLineage{E} cannot exceed \existedset{S_n} in every step for sequences in \lineage{\potacc{S_0}}, as it relies on \potacc{} and \ProjPotAcc{}, both of which do not consider non-existent \TMobjs{}.
Theorem \ref{SW:thm:mutableLineageExisted} captures this invariant formally in \Cref{fig:safety:agNodesSpecEndow} and corresponds to \COQagNodesSpecEndow{} theorem in \TMmodelName{}.
Presently, it is a useful diagnostic for determining if the erroneous cases of SW \thmauthred{} have been fixed.
It will be specialized for precision in the theorem in \Cref{SW:thm:mutableLineage:singleLin}.

\subsection{Main Theorem}
\label{SW:sec:maintheorem}

\begin{figure}
  \begin{tabular}{l}
    If \multiop{S}{n} is an execution, \\
    and \(\multiop{\AccessSeq{P}}{n} \rightsleq \multiopfree{\AccessSeq{L}}{n} \in \lineage{\potacc{S_0}}\), \\
    and \(E \subseteq \existedset{S_{i-1}}\), \\
  \end{tabular}
  \[
  \begin{array}{ll}
    \multicolumn{2}{l}{\mutableLineageApp{E}{\singleopfree{\AccessSeqElem{L}{i-1}}{\AccessSeqElem{L}{i}}} \subseteq} \\
     \hspace{10mm} & \left\{ 
    \begin{array}{rl}
      E \cup \{u\} & \textnormal{if} \; \AccessSeqElem{L}{i} = \ProjPotAccB{\AccessSeqElem{L}{i-1}}{p}{u} \land p \in E \\
      E & \textnormal{if } \AccessSeqElem{L}{i} = \ProjPotAccB{\AccessSeqElem{L}{i-1}}{p}{u} \land p \notin E \\
      E & \textnormal{if } \AccessSeqElem{L}{i} = \AccessSeqElem{L}{i-1}
    \end{array}
    \right.
  \end{array}
  \]
  \caption{Theorem: \mutableLineage{} grows only by new \TMobjs{} at each step. \label{SW:thm:mutableLineage:singleLin}}
\end{figure}

There are two other theorems required to successfully prove the main theorem.  
Consider how \mutableLineage{} changes over sequences in \lineage{R}.  
Each stepin \lineage{R} is given by some \ProjPotAccB{\AccessRel{A}}{p}{u}.
If \(p \in E\), then \mutableLineage{} can only grown by u, otherwise \mutableLineage{} is constant.
Another way of stating this is that \ProjPotAcc{} is surjective on set inclusion.
As the only objects added to \mutableLineage{} are not members of \existedset{S_0}, they are excluded when intersecting the result with what initially existed.
As the base case of \mutableLineage{} is \mutable{}, this induction must hold.
Although these properties are many individual theorems in \TMmodelName{}, they are presented as a single property in \Cref{SW:thm:mutableLineage:singleLin}.

\begin{figure}
  \begin{tabular}{l}
    If \(\multiop{S}{n}\) is an execution, \\
    and \(\multiop{\AccessSeq{P}}{n} \leq \multiopfree{\AccessSeq{L}}{n} \in \lineage{\potacc{S_0}}\), \\
    and \(E \subseteq \existedset{S_i}\), \\
    and \( l_i = \multiopfree{\AccessSeq{L}}{i}\) are sub-sequences
  \end{tabular}
\[
\mutableLineageApp{E}{l_n} \cap \existedset{S_0} \subseteq \mutableapp{\potacc{S_0}}{E}
\]
\caption{Theorem: \mutableLineage{} is \mutable{}. \label{SW:thm:mutableLineage:Lin}}
\end{figure}

The \(E \subseteq \existedset{S_i}\) requirement is an invariant of \mutableLineage{} over \lineage{}, so that constraint can be satisfied.
The remaining cases are evident by examination of the theorem in \Cref{thm:SW:ProjPotAccOp}.
Using the result in \Cref{SW:thm:mutableLineage:singleLin} as an induction step, the theorem in \Cref{SW:thm:mutableLineage:Lin} is produced.

\begin{figure}
  \begin{tabular}{l}
    If \(\multiop{S}{n}\) is an execution,\\
    and \(E \subseteq \existedset{S_i}\),\\
    and \( e_i = \multiop{S}{i}\) are sub-executions
    \end{tabular}
  \[
  \mutatedapp{\potacc{S_0}}{e_n} \cap \existedset{S_0} \subseteq \mutableapp{\potacc{S_0}}{E}
  \]
  Proof:
  \[
  \begin{array}{rl}
    %left
    \mutatedapp{\potacc{S_0}}{e_n} \cap S_0^{existed} & \subseteq \mutableapp{\potacc{S_0}}{E} \\
    \multicolumn{2}{l}{\textnormal{let} \; p_i = \multiop{\AccessSeq{P}}{i} \textnormal{in}} \\
    %right
    \multicolumn{2}{r}{\textnormal{by Definition}} \\
    %left
    \mutableLineageApp{E}{p_n} \cap \existedset{S_0} & \subseteq \mutableapp{\potacc{S_0}}{E} \\
    %right
    \multicolumn{2}{r}{\textnormal{by Theorem \ref{thm:SW:mutatedmutableLinage} instantiated reflexively}} \\
    %left
    \multicolumn{2}{l}{\textnormal{let} \; \multiop{\AccessSeq{P}}{n} \rightsleq \multiopfree{\AccessSeq{L}}{n} \in \lineage{\potacc{S_0}}} \\
    \multicolumn{2}{l}{\textnormal{and} \; l_i = \multiopfree{\AccessSeq{L}}{i} \; \textnormal{in}} \\
    %right
    \multicolumn{2}{r}{\textnormal{by Theorem \ref{thm:SW:ProjPotAccApprox}}} \\
    %left
    \mutableLineageApp{E}{l_n} \cap \existedset{S_0} & \subseteq \mutableapp{\potacc{S_0}}{E} \\
    %right
    \multicolumn{2}{r}{\textnormal{by Theorem \ref{SW:thm:mutableLineageRespectful}}} \\
    \multicolumn{2}{r}{\textnormal{this is Theorem \ref{SW:thm:mutableLineage:Lin}}}
  \end{array}
  \]
  \caption{Corrected final theorem and proof. \label{SW:thm:correctedMain}}
\end{figure}

With the exception of a parameterized definition of \mutable{}, the statement of the main theorem in \Cref{SW:thm:correctedMain} is identical to the \TMSW{} theorem.
Discharging the main theorem proceeds by subset transitivity and previous definitions.
From \Cref{thm:SW:mutatedmutableLinage}, \mutableLineage{} approximates \mutated{} and may be substituted by transitivity.
There must exist a lineage in \(\lineage{\potacc{S_0}}\) that approximates \potacc{}.
Because \mutableLineage{} preserves ordering in \Cref{SW:thm:mutableLineageRespectful}, it must also approximate \potacc{} and may be substituted by transitivity.
The remaining relation is precisely that \mutableLineage{} is \mutable{} from \Cref{SW:thm:mutableLineage:Lin}.
Therefore, the main theorem holds; what is \mutated{}, when restricted to what {\bf existed}, is a subset of what was determined \mutable{}.

As \TMmodelName{} began as a mechanical verification attempt for \TMSW{}, it can be used to correct the proof execution.
The corrections in this chapter are altered from those found in \TMmodelName{}, but follow the spirit of that proof.
They also contain some of the suggestions and simplifications found in \Cref{ch:future} on future work.
These alterations generally impact the proof execution, but do not significantly impact the statement of the main theorem in \TMSW{}.
Previous conclusions based on the \TMSW{} result \cite{Shapiro03thepractical} can be expected to hold without modification.

\endgroup
